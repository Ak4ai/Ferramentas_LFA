% !TEX program = pdflatex
\documentclass[a4paper,12pt]{article}

% ================== PACOTES ==================
\usepackage[utf8]{inputenc}
\usepackage[T1]{fontenc}
\usepackage[brazilian]{babel}
\usepackage{amsmath,amssymb,amsthm}
\usepackage{graphicx}
\usepackage{float}
\usepackage{hyperref}
\usepackage{fancyhdr}
\usepackage{geometry}
\usepackage{tcolorbox}
\usepackage{enumitem}
\usepackage{booktabs}
\usepackage{multirow}
\usepackage{array}
\usepackage{xcolor}
\usepackage{tikz}

% ================== CONFIGURAÇÕES ==================
\geometry{
    left=2.5cm,
    right=2.5cm,
    top=2.5cm,
    bottom=2.5cm
}

\hypersetup{
    colorlinks=true,
    linkcolor=blue,
    filecolor=magenta,
    urlcolor=cyan,
}

% ================== CABEÇALHO/RODAPÉ ==================
\pagestyle{fancy}
\fancyhf{}
\fancyhead[L]{LFA - Linguagens Formais e Autômatos}
\fancyhead[R]{CEFET-MG}
\fancyfoot[C]{\thepage}
\renewcommand{\headrulewidth}{0.4pt}
\renewcommand{\footrulewidth}{0.4pt}

% ================== BOXES ==================
\newtcolorbox{enunciadobox}[1][]{
    colback=blue!5,
    colframe=blue!60!black,
    title={\textbf{#1}},
    fonttitle=\bfseries
}

\newtcolorbox{explicacaobox}[1][]{
    colback=green!5,
    colframe=green!50!black,
    title={\textbf{#1}},
    fonttitle=\bfseries
}

% ================== TÍTULO ==================
\title{\textbf{Lista de Exercícios}\\
\Large Autômatos de Pilha (AP)\\
\large Resoluções com Diagramas}
\author{Linguagens Formais e Autômatos}
\date{\today}

\begin{document}

\maketitle
\tableofcontents
\newpage

% ============================================================
% EXEMPLO: a^n b^n
% ============================================================
\section{Exemplo: Linguagem $a^n b^n$}

\begin{enunciadobox}[Enunciado]
Construa um AP que reconheça a linguagem $L = \{a^n b^n \mid n \geq 0\}$
\end{enunciadobox}

\begin{explicacaobox}[Estratégia]
Empilha um símbolo para cada $a$ lido, depois desempilha um para cada $b$. Aceita se a pilha ficar vazia (apenas $Z$) ao final.
\end{explicacaobox}

\begin{figure}[H]
    \centering
    \includegraphics[width=\textwidth,height=0.4\textheight,keepaspectratio]{../diagramasAP/AP_anbn_igual.pdf}
    \caption{AP Exemplo: $L = \{a^n b^n\}$}
\end{figure}

\newpage

% ============================================================
% EXERCÍCIO 1
% ============================================================
\section{Exercício 1: Linguagem Vazia}

\begin{enunciadobox}[Enunciado]
Construa um AP que reconheça a seguinte linguagem: $L = \emptyset$
\end{enunciadobox}

\begin{explicacaobox}[Estratégia]
Um AP que reconhece a linguagem vazia não possui nenhum estado final acessível. Qualquer entrada será rejeitada.
\end{explicacaobox}

\begin{figure}[H]
    \centering
    \includegraphics[width=\textwidth,height=0.4\textheight,keepaspectratio]{../diagramasAP/AP_exe1_vazio.pdf}
    \caption{AP 1: $L = \emptyset$}
\end{figure}

\newpage

% ============================================================
% EXERCÍCIO 2
% ============================================================
\section{Exercício 2: Linguagens sobre $\Sigma = \{a, b, c\}$}

% ============================================================
% EXERCÍCIO 2a
% ============================================================
\subsection{Exercício 2a: Palavra Vazia}

\begin{enunciadobox}[Enunciado]
$L = \{w \mid w = \lambda\}$ (apenas a palavra vazia)
\end{enunciadobox}

\begin{figure}[H]
    \centering
    \includegraphics[width=\textwidth,height=0.4\textheight,keepaspectratio]{../diagramasAP/AP_exe2a_lambda.pdf}
    \caption{AP 2a: $L = \{\lambda\}$}
\end{figure}

\newpage

% ============================================================
% EXERCÍCIO 2b
% ============================================================
\subsection{Exercício 2b: $i > j$}

\begin{enunciadobox}[Enunciado]
$L = \{a^i b^j \mid i > j\}$
\end{enunciadobox}

\begin{explicacaobox}[Estratégia]
Empilha $A$ para cada $a$, desempilha para cada $b$. Aceita se ainda houver $A$'s na pilha ao final (mais $a$'s que $b$'s).
\end{explicacaobox}

\begin{figure}[H]
    \centering
    \includegraphics[width=\textwidth,height=0.4\textheight,keepaspectratio]{../diagramasAP/AP_exe2b_anbn_imaiorj.pdf}
    \caption{AP 2b: $L = \{a^i b^j \mid i > j\}$}
\end{figure}

\newpage

% ============================================================
% EXERCÍCIO 2c
% ============================================================
\subsection{Exercício 2c: $i < j$}

\begin{enunciadobox}[Enunciado]
$L = \{a^i b^j \mid i < j\}$
\end{enunciadobox}

\begin{explicacaobox}[Estratégia]
Empilha $A$ para cada $a$, desempilha para cada $b$. Após consumir todos os $a$'s, continua aceitando $b$'s extras (empilhando $B$'s). Aceita se houver $B$'s na pilha ao final.
\end{explicacaobox}

\begin{figure}[H]
    \centering
    \includegraphics[width=\textwidth,height=0.4\textheight,keepaspectratio]{../diagramasAP/AP_exe2c_anbn_imenorj.pdf}
    \caption{AP 2c: $L = \{a^i b^j \mid i < j\}$}
\end{figure}

\newpage

% ============================================================
% EXERCÍCIO 2d
% ============================================================
\subsection{Exercício 2d: $i \neq j$}

\begin{enunciadobox}[Enunciado]
$L = \{a^i b^j \mid i \neq j\}$
\end{enunciadobox}

\begin{explicacaobox}[Estratégia]
União de $i > j$ e $i < j$. O AP não-determinístico escolhe qual ramo seguir.
\end{explicacaobox}

\begin{figure}[H]
    \centering
    \includegraphics[width=\textwidth,height=0.4\textheight,keepaspectratio]{../diagramasAP/AP_exe2d_anbn_idiferentej.pdf}
    \caption{AP 2d: $L = \{a^i b^j \mid i \neq j\}$}
\end{figure}

\newpage

% ============================================================
% EXERCÍCIO 2e
% ============================================================
\subsection{Exercício 2e: $wcw^R$}

\begin{enunciadobox}[Enunciado]
$L = \{wcw^R \mid w \in \{a, b\}^*\}$
\end{enunciadobox}

\begin{explicacaobox}[Estratégia]
Empilha todos os símbolos antes do $c$. Após ler $c$, desempilha comparando com os símbolos restantes. Aceita se tudo casar (palíndromo com marcador central).
\end{explicacaobox}

\begin{figure}[H]
    \centering
    \includegraphics[width=\textwidth,height=0.4\textheight,keepaspectratio]{../diagramasAP/AP_exe2e_wcw_reverse.pdf}
    \caption{AP 2e: $L = \{wcw^R\}$}
\end{figure}

\newpage

% ============================================================
% EXERCÍCIO 2f
% ============================================================
\subsection{Exercício 2f: $k = 2(i + j)$}

\begin{enunciadobox}[Enunciado]
$L = \{a^i b^j c^k \mid i, j, k > 0 \text{ e } k = 2(i + j)\}$
\end{enunciadobox}

\begin{explicacaobox}[Estratégia]
Para cada $a$ ou $b$, empilha 2 símbolos (representando que cada um contribui 2 para o total de $c$'s necessários). Depois desempilha um para cada $c$.
\end{explicacaobox}

\begin{figure}[H]
    \centering
    \includegraphics[width=\textwidth,height=0.4\textheight,keepaspectratio]{../diagramasAP/AP_exe2f_k2ij.pdf}
    \caption{AP 2f: $L = \{a^i b^j c^k \mid k = 2(i+j)\}$}
\end{figure}

\newpage

% ============================================================
% EXERCÍCIO 2g
% ============================================================
\subsection{Exercício 2g: $i = j$ ou $j = k$}

\begin{enunciadobox}[Enunciado]
$L = \{a^i b^j c^k \mid i, j, k > 0 \text{ e } (i = j \text{ ou } j = k)\}$
\end{enunciadobox}

\begin{explicacaobox}[Estratégia]
AP não-determinístico com duas ramificações:
\begin{itemize}
    \item Ramo 1 ($i = j$): Empilha para cada $a$, desempilha para cada $b$, ignora $c$'s
    \item Ramo 2 ($j = k$): Ignora $a$'s, empilha para cada $b$, desempilha para cada $c$
\end{itemize}
\end{explicacaobox}

\begin{figure}[H]
    \centering
    \includegraphics[width=\textwidth,height=0.4\textheight,keepaspectratio]{../diagramasAP/AP_exe2g_ij_ou_jk.pdf}
    \caption{AP 2g: $L = \{a^i b^j c^k \mid i = j \lor j = k\}$}
\end{figure}

\newpage

% ============================================================
% EXERCÍCIO 2h
% ============================================================
\subsection{Exercício 2h: $a^* w c^n$}

\begin{enunciadobox}[Enunciado]
$L = \{a^* w c^n \mid w \in \{a, b\}^* \text{ e } n = |w|_a\}$

(Número de $c$'s igual ao número de $a$'s em $w$)
\end{enunciadobox}

\begin{explicacaobox}[Estratégia]
O prefixo $a^*$ é ignorado (não conta). Para a parte $w$, empilha um símbolo para cada $a$ encontrado (ignora $b$'s). Depois desempilha um para cada $c$.
\end{explicacaobox}

\begin{figure}[H]
    \centering
    \includegraphics[width=\textwidth,height=0.4\textheight,keepaspectratio]{../diagramasAP/AP_exe2h_a_w_cn.pdf}
    \caption{AP 2h: $L = \{a^* w c^n \mid n = |w|_a\}$}
\end{figure}

\newpage

% ============================================================
% EXERCÍCIO 3
% ============================================================
\section{Exercício 3: APD para $a^n b^n c^m d^m$}

\begin{enunciadobox}[Enunciado]
Construa um APD que reconheça a seguinte linguagem:

$L = \{a^n b^n c^m d^m \mid n, m \in \mathbb{N}\}$
\end{enunciadobox}

\begin{explicacaobox}[Estratégia]
O autômato processa em duas fases independentes:
\begin{enumerate}
    \item Empilha para cada $a$, desempilha para cada $b$ (garante $n$ $a$'s e $n$ $b$'s)
    \item Empilha para cada $c$, desempilha para cada $d$ (garante $m$ $c$'s e $m$ $d$'s)
\end{enumerate}
Como as fases são independentes, é determinístico.
\end{explicacaobox}

\begin{figure}[H]
    \centering
    \includegraphics[width=\textwidth,height=0.4\textheight,keepaspectratio]{../diagramasAP/AP_exe3_anbncmdm.pdf}
    \caption{APD 3: $L = \{a^n b^n c^m d^m\}$}
\end{figure}

\newpage

% ============================================================
% EXERCÍCIO 4
% ============================================================
\section{Exercício 4: APD para $i + j = k + l$}

\begin{enunciadobox}[Enunciado]
Construa um APD que reconheça a seguinte linguagem:

$L = \{a^i b^j c^k d^l \mid i + j = k + l \text{ e } i, j, k, l \in \mathbb{N}\}$
\end{enunciadobox}

\begin{explicacaobox}[Estratégia]
Empilha um símbolo para cada $a$ e cada $b$ (acumulando $i + j$). Depois desempilha um para cada $c$ e cada $d$ (consumindo $k + l$). Aceita se pilha voltar ao estado inicial.
\end{explicacaobox}

\begin{figure}[H]
    \centering
    \includegraphics[width=\textwidth,height=0.4\textheight,keepaspectratio]{../diagramasAP/AP_exe4_ijkl_soma.pdf}
    \caption{APD 4: $L = \{a^i b^j c^k d^l \mid i+j=k+l\}$}
\end{figure}

\newpage

% ============================================================
% EXERCÍCIO 5
% ============================================================
\section{Exercício 5: AP para $i + j = k + l$ (com $a$ final)}

\begin{enunciadobox}[Enunciado]
Construa um AP que reconheça a seguinte linguagem:

$L = \{a^i b^j c^k a^l \mid i + j = k + l \text{ e } i, j, k, l \in \mathbb{N}\}$
\end{enunciadobox}

\begin{explicacaobox}[Estratégia]
Similar ao exercício 4, mas o último símbolo é $a$ em vez de $d$. Empilha para $a$'s e $b$'s iniciais, desempilha para $c$'s e $a$'s finais.
\end{explicacaobox}

\begin{figure}[H]
    \centering
    \includegraphics[width=\textwidth,height=0.4\textheight,keepaspectratio]{../diagramasAP/AP_exe5_ijkl_soma_a.pdf}
    \caption{AP 5: $L = \{a^i b^j c^k a^l \mid i+j=k+l\}$}
\end{figure}

\newpage

% ============================================================
% EXERCÍCIO 6
% ============================================================
\section{Exercício 6: Linguagens Regulares e Livres de Contexto}

% ============================================================
% EXERCÍCIO 6a
% ============================================================
\subsection{Exercício 6a: Pelo menos três 1's}

\begin{enunciadobox}[Enunciado]
$L = \{w \in \{0, 1\}^* \mid w \text{ possui pelo menos três } 1\text{'s}\}$
\end{enunciadobox}

\begin{explicacaobox}[Observação]
Esta é uma linguagem regular! A pilha não é utilizada. O AP funciona como um AFD, contando os 1's com os estados.
\end{explicacaobox}

\begin{figure}[H]
    \centering
    \includegraphics[width=\textwidth,height=0.4\textheight,keepaspectratio]{../diagramasAP/AP_exe6a_3uns.pdf}
    \caption{AP 6a: Pelo menos três 1's}
\end{figure}

\newpage

% ============================================================
% EXERCÍCIO 6b
% ============================================================
\subsection{Exercício 6b: Quantidade ímpar de 0's}

\begin{enunciadobox}[Enunciado]
$L = \{w \in \{0, 1\}^* \mid w \text{ possui quantidade ímpar de } 0\text{'s}\}$
\end{enunciadobox}

\begin{explicacaobox}[Observação]
Linguagem regular. O AP alterna entre dois estados a cada 0 lido, sem usar a pilha.
\end{explicacaobox}

\begin{figure}[H]
    \centering
    \includegraphics[width=\textwidth,height=0.4\textheight,keepaspectratio]{../diagramasAP/AP_exe6b_impar0s.pdf}
    \caption{AP 6b: Quantidade ímpar de 0's}
\end{figure}

\newpage

% ============================================================
% EXERCÍCIO 6c
% ============================================================
\subsection{Exercício 6c: $\Sigma^*$}

\begin{enunciadobox}[Enunciado]
$L = \{w \in \{0, 1\}^* \mid w \text{ pode ser descrito como a ER } (0^* 1^*)^*\}$
\end{enunciadobox}

\begin{explicacaobox}[Observação]
$(0^* 1^*)^* = \Sigma^*$ (todas as cadeias sobre $\{0, 1\}$). O AP aceita qualquer entrada sem usar a pilha.
\end{explicacaobox}

\begin{figure}[H]
    \centering
    \includegraphics[width=\textwidth,height=0.4\textheight,keepaspectratio]{../diagramasAP/AP_exe6c_sigma_star.pdf}
    \caption{AP 6c: $\Sigma^*$}
\end{figure}

\newpage

% ============================================================
% EXERCÍCIO 6d
% ============================================================
\subsection{Exercício 6d: Parênteses Balanceados}

\begin{enunciadobox}[Enunciado]
$L = \{w \mid w \text{ é formado por parênteses balanceados}\}$

Exemplo: \texttt{( ( ) ) ( ) ( ( ) )}
\end{enunciadobox}

\begin{explicacaobox}[Estratégia]
Empilha um símbolo para cada \texttt{(}, desempilha para cada \texttt{)}. Aceita se a pilha estiver vazia ao final e nunca ficar ``negativa''.
\end{explicacaobox}

\begin{figure}[H]
    \centering
    \includegraphics[width=\textwidth,height=0.4\textheight,keepaspectratio]{../diagramasAP/AP_exe6d_parenteses.pdf}
    \caption{AP 6d: Parênteses Balanceados}
\end{figure}

\newpage

% ============================================================
% EXERCÍCIO 6e
% ============================================================
\subsection{Exercício 6e: Parênteses e Colchetes Balanceados}

\begin{enunciadobox}[Enunciado]
$L = \{w \mid w \text{ é formado por parênteses e colchetes balanceados}\}$

Exemplo: \texttt{[ ( [ ( ) ] ) ( ) ] ( ( ) [ ( ) ] )}
\end{enunciadobox}

\begin{explicacaobox}[Estratégia]
Empilha símbolos diferentes para \texttt{(} e \texttt{[}. Ao ler \texttt{)} ou \texttt{]}, verifica se o topo da pilha corresponde ao tipo correto de abertura.
\end{explicacaobox}

\begin{figure}[H]
    \centering
    \includegraphics[width=\textwidth,height=0.4\textheight,keepaspectratio]{../diagramasAP/AP_exe6e_parenteses_colchetes.pdf}
    \caption{AP 6e: Parênteses e Colchetes Balanceados}
\end{figure}

\newpage

% ============================================================
% EXERCÍCIO 6f
% ============================================================
\subsection{Exercício 6f: $a^n b^n \cup b^n a^n$}

\begin{enunciadobox}[Enunciado]
$L = \{w \mid w \text{ é formado por } n \text{ } a\text{'s seguidos de } n \text{ } b\text{'s, ou } n \text{ } b\text{'s seguidos de } n \text{ } a\text{'s}\}$
\end{enunciadobox}

\begin{explicacaobox}[Estratégia]
AP não-determinístico que escolhe entre duas ramificações:
\begin{itemize}
    \item Ramo 1: Empilha $a$'s, desempilha com $b$'s
    \item Ramo 2: Empilha $b$'s, desempilha com $a$'s
\end{itemize}
\end{explicacaobox}

\begin{figure}[H]
    \centering
    \includegraphics[width=\textwidth,height=0.4\textheight,keepaspectratio]{../diagramasAP/AP_exe6f_anbn_ou_bnan.pdf}
    \caption{AP 6f: $a^n b^n \cup b^n a^n$}
\end{figure}

\newpage

% ============================================================
% EXERCÍCIO 6g
% ============================================================
\subsection{Exercício 6g: Palíndromos de Tamanho Ímpar}

\begin{enunciadobox}[Enunciado]
$L = \{w \in \{0, 1\}^* \mid w = w^R \text{ e } |w| \text{ é ímpar}\}$
\end{enunciadobox}

\begin{explicacaobox}[Estratégia]
Empilha símbolos até não-deterministicamente decidir que está no meio (lendo o símbolo central). Depois desempilha verificando se os símbolos casam.
\end{explicacaobox}

\begin{figure}[H]
    \centering
    \includegraphics[width=\textwidth,height=0.4\textheight,keepaspectratio]{../diagramasAP/AP_exe6g_palindromo_impar.pdf}
    \caption{AP 6g: Palíndromos de Tamanho Ímpar}
\end{figure}

\newpage

% ============================================================
% EXERCÍCIO 6h
% ============================================================
\subsection{Exercício 6h: Palíndromos}

\begin{enunciadobox}[Enunciado]
$L = \{w \in \{0, 1\}^* \mid w = w^R\}$ (todos os palíndromos)
\end{enunciadobox}

\begin{explicacaobox}[Estratégia]
Similar ao 6g, mas também aceita palíndromos de tamanho par. O AP não-deterministicamente escolhe quando está no meio:
\begin{itemize}
    \item Para tamanho par: transição $\varepsilon$ para fase de desempilhamento
    \item Para tamanho ímpar: lê símbolo central sem alterar pilha
\end{itemize}
\end{explicacaobox}

\begin{figure}[H]
    \centering
    \includegraphics[width=\textwidth,height=0.4\textheight,keepaspectratio]{../diagramasAP/AP_exe6h_palindromo.pdf}
    \caption{AP 6h: Palíndromos}
\end{figure}

\newpage

% ============================================================
% RESUMO
% ============================================================
\section{Resumo dos Exercícios}

\begin{center}
\begin{tabular}{|c|l|c|c|}
\hline
\textbf{Ex.} & \textbf{Linguagem} & \textbf{Alfabeto} & \textbf{Tipo} \\
\hline
Ex & $a^n b^n$ & $\{a,b\}$ & LLC \\
1 & $L = \emptyset$ & $\{a,b,c\}$ & - \\
2a & $L = \{\lambda\}$ & $\{a,b,c\}$ & Regular \\
2b & $a^i b^j$, $i > j$ & $\{a,b,c\}$ & LLC \\
2c & $a^i b^j$, $i < j$ & $\{a,b,c\}$ & LLC \\
2d & $a^i b^j$, $i \neq j$ & $\{a,b,c\}$ & LLC \\
2e & $wcw^R$ & $\{a,b,c\}$ & LLC \\
2f & $a^i b^j c^k$, $k = 2(i+j)$ & $\{a,b,c\}$ & LLC \\
2g & $a^i b^j c^k$, $i=j \lor j=k$ & $\{a,b,c\}$ & LLC \\
2h & $a^* w c^n$, $n = |w|_a$ & $\{a,b,c\}$ & LLC \\
\hline
3 & $a^n b^n c^m d^m$ & $\{a,b,c,d\}$ & LLC (APD) \\
4 & $a^i b^j c^k d^l$, $i+j=k+l$ & $\{a,b,c,d\}$ & LLC (APD) \\
5 & $a^i b^j c^k a^l$, $i+j=k+l$ & $\{a,b,c\}$ & LLC \\
\hline
6a & Pelo menos 3 uns & $\{0,1\}$ & Regular \\
6b & Quantidade ímpar de 0's & $\{0,1\}$ & Regular \\
6c & $\Sigma^*$ & $\{0,1\}$ & Regular \\
6d & Parênteses balanceados & $\{(,)\}$ & LLC (APD) \\
6e & Parênteses e colchetes balanceados & $\{(,),[,]\}$ & LLC (APD) \\
6f & $a^n b^n \cup b^n a^n$ & $\{a,b\}$ & LLC \\
6g & Palíndromos ímpares & $\{0,1\}$ & LLC \\
6h & Palíndromos & $\{0,1\}$ & LLC \\
\hline
\end{tabular}
\end{center}

\vspace{1cm}

\textbf{Legenda:}
\begin{itemize}
    \item \textbf{LLC}: Linguagem Livre de Contexto (requer pilha)
    \item \textbf{APD}: Autômato de Pilha Determinístico
    \item \textbf{Regular}: Linguagem regular (pilha não é necessária)
\end{itemize}

\section*{Observações}

\begin{itemize}
    \item Todos os APs foram validados com testes aleatórios
    \item Os diagramas foram gerados automaticamente usando Mermaid
    \item Os arquivos JSON de definição estão em \texttt{inputAP/}
    \item Os diagramas PDF estão em \texttt{diagramasAP/}
    \item Notação de transições: $\sigma, X/\alpha$ significa ``lendo $\sigma$, com $X$ no topo, substitui por $\alpha$''
    \item $\varepsilon$ representa transição vazia (sem consumir entrada)
\end{itemize}

\end{document}
