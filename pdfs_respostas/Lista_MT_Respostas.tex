\documentclass[12pt,a4paper]{article}
\usepackage{fontspec}  % LuaLaTeX - suporte a fontes
\usepackage[brazil]{babel}
\usepackage{amsmath,amssymb,amsfonts}
\usepackage{graphicx}
\usepackage{geometry}

% Caracteres Unicode funcionam nativamente com LuaLaTeX
\usepackage{booktabs}
\usepackage{array}
\usepackage{longtable}
\usepackage{fancyhdr}
\usepackage{hyperref}
\usepackage{xcolor}
\usepackage{tcolorbox}
\usepackage{listings}
\usepackage{enumitem}
\usepackage{float}

\geometry{
    left=2.5cm,
    right=2.5cm,
    top=2.5cm,
    bottom=2.5cm
}

\pagestyle{fancy}
\fancyhf{}
\rhead{Linguagens Formais e Autômatos}
\lhead{Lista de Máquinas de Turing}
\rfoot{Página \thepage}

\definecolor{accept}{RGB}{144, 238, 144}
\definecolor{reject}{RGB}{255, 182, 193}
\definecolor{codeblue}{RGB}{0, 102, 204}

\tcbuselibrary{skins,breakable}

\newtcolorbox{acceptbox}{
    colback=accept!30,
    colframe=green!50!black,
    title={\textbf{ACEITA}},
    fonttitle=\bfseries
}

\newtcolorbox{rejectbox}{
    colback=reject!30,
    colframe=red!50!black,
    title={\textbf{REJEITA}},
    fonttitle=\bfseries
}

\title{
    \vspace{-2cm}
    \textbf{Lista de Exercícios}\\
    \Large{Máquinas de Turing}\\
    \vspace{0.5cm}
    \normalsize{Linguagens Formais e Autômatos}
}
\author{CEFET}
\date{Dezembro 2025}

\begin{document}

\maketitle
\tableofcontents
\newpage

%%%%%%%%%%%%%%%%%%%%%%%%%%%%%%%%%%%%%%%%%%%%%%%%%%%%%%%%%%%%%%%%%%%%%%%%%%%%%%%
\section{Exercício 1: Construção de Máquinas de Turing}
%%%%%%%%%%%%%%%%%%%%%%%%%%%%%%%%%%%%%%%%%%%%%%%%%%%%%%%%%%%%%%%%%%%%%%%%%%%%%%%

\textbf{Enunciado:} Construa Máquinas de Turing padrão que aceitem as seguintes linguagens:

%------------------------------------------------------------------------------
\subsection{1.a) $L = \{w \mid w \in \{a, b, c\}^* \text{ e } w \text{ começa com } ab\}$}
%------------------------------------------------------------------------------

\subsubsection*{Descrição}
Esta MT verifica se a cadeia de entrada começa com os símbolos $ab$ em sequência.

\subsubsection*{Algoritmo}
\begin{enumerate}
    \item Verifica se o primeiro símbolo é $a$
    \item Move para a direita e verifica se o segundo símbolo é $b$
    \item Se ambos estiverem corretos, aceita
    \item Caso contrário, rejeita
\end{enumerate}

\subsubsection*{Diagrama de Estados}
\begin{figure}[H]
    \centering
    \includegraphics[width=0.8\textwidth,height=0.4\textheight,keepaspectratio]{diagramas/MT_exe1_a.pdf}
    \caption{MT para $L = \{w \mid w \text{ começa com } ab\}$}
\end{figure}

\newpage
%------------------------------------------------------------------------------
\subsection{1.b) $L = \{a^n b^n c^n \mid n \geq 0\}$}
%------------------------------------------------------------------------------

\subsubsection*{Descrição}
Esta MT aceita cadeias com igual quantidade de $a$'s, $b$'s e $c$'s em sequência.

\subsubsection*{Algoritmo}
\begin{enumerate}
    \item Marca um $a$ com $X$
    \item Busca e marca um $b$ com $Y$
    \item Busca e marca um $c$ com $Z$
    \item Retorna ao início e repete até não haver mais símbolos não marcados
    \item Se sobrar algum símbolo não pareado, rejeita
\end{enumerate}

\subsubsection*{Diagrama de Estados}
\begin{figure}[H]
    \centering
    \includegraphics[width=0.8\textwidth,height=0.4\textheight,keepaspectratio]{diagramas/MT_exe1_b.pdf}
    \caption{MT para $L = \{a^n b^n c^n \mid n \geq 0\}$}
\end{figure}

\newpage
%------------------------------------------------------------------------------
\subsection{1.c) $L = \{a^n b^m \mid n, m \geq 0 \text{ e } n = 2m\}$}
%------------------------------------------------------------------------------

\subsubsection*{Descrição}
Esta MT aceita cadeias onde o número de $a$'s é o dobro do número de $b$'s.

\subsubsection*{Algoritmo}
\begin{enumerate}
    \item Marca dois $a$'s com $X$
    \item Busca e marca um $b$ com $Y$
    \item Retorna e repete
    \item Aceita se todos os símbolos forem pareados corretamente
\end{enumerate}

\subsubsection*{Diagrama de Estados}
\begin{figure}[H]
    \centering
    \includegraphics[width=0.8\textwidth,height=0.4\textheight,keepaspectratio]{diagramas/MT_exe1_c.pdf}
    \caption{MT para $L = \{a^n b^m \mid n = 2m\}$}
\end{figure}

\newpage
%------------------------------------------------------------------------------
\subsection{1.d) $L = \{ww^R \mid w \in \{a, b\}^*\}$}
%------------------------------------------------------------------------------

\subsubsection*{Descrição}
Esta MT aceita palíndromos pares, ou seja, cadeias que são iguais lidas da esquerda para a direita e da direita para a esquerda.

\subsubsection*{Algoritmo}
\begin{enumerate}
    \item Memoriza e marca o primeiro símbolo
    \item Vai até o último símbolo não marcado
    \item Verifica se é igual ao memorizado
    \item Marca e retorna ao início
    \item Repete até todos estarem marcados
\end{enumerate}

\subsubsection*{Diagrama de Estados}
\begin{figure}[H]
    \centering
    \includegraphics[width=0.8\textwidth,height=0.4\textheight,keepaspectratio]{diagramas/MT_exe1_d.pdf}
    \caption{MT para $L = \{ww^R\}$ (palíndromos pares)}
\end{figure}

\newpage
%------------------------------------------------------------------------------
\subsection{1.e) $L = \{w \mid w \in \{a, b\}^* \text{ e } n_a(w) = n_b(w)\}$}
%------------------------------------------------------------------------------

\subsubsection*{Descrição}
Esta MT aceita cadeias com igual quantidade de $a$'s e $b$'s, em qualquer ordem.

\subsubsection*{Algoritmo}
\begin{enumerate}
    \item Encontra um $a$ não marcado, marca com $X$
    \item Busca um $b$ não marcado, marca com $X$
    \item (Ou vice-versa: encontra $b$ primeiro, depois $a$)
    \item Retorna ao início e repete
    \item Aceita se todos os símbolos forem pareados
\end{enumerate}

\subsubsection*{Diagrama de Estados}
\begin{figure}[H]
    \centering
    \includegraphics[width=0.8\textwidth,height=0.4\textheight,keepaspectratio]{diagramas/MT_exe1_e.pdf}
    \caption{MT para $L = \{w \mid n_a(w) = n_b(w)\}$}
\end{figure}

\newpage
%------------------------------------------------------------------------------
\subsection{1.f) $L = \{1^n 0^{n+3} \mid n \geq 0\}$}
%------------------------------------------------------------------------------

\subsubsection*{Descrição}
Esta MT aceita cadeias com $n$ uns seguidos de $n+3$ zeros.

\subsubsection*{Algoritmo}
\begin{enumerate}
    \item Verifica que existem pelo menos 3 zeros no início (após os 1's)
    \item Para cada $1$, marca e busca um $0$ correspondente
    \item Aceita se a contagem estiver correta
\end{enumerate}

\subsubsection*{Diagrama de Estados}
\begin{figure}[H]
    \centering
    \includegraphics[width=0.8\textwidth,height=0.4\textheight,keepaspectratio]{diagramas/MT_exe1_f.pdf}
    \caption{MT para $L = \{1^n 0^{n+3}\}$}
\end{figure}

\newpage
%------------------------------------------------------------------------------
\subsection{1.g) $L = \{a^n b^{2n} c^{n-1} \mid n > 0\}$}
%------------------------------------------------------------------------------

\subsubsection*{Descrição}
Esta MT aceita cadeias onde:
\begin{itemize}
    \item Número de $b$'s é o dobro do número de $a$'s
    \item Número de $c$'s é um a menos que o número de $a$'s
\end{itemize}

\subsubsection*{Algoritmo}
\begin{enumerate}
    \item Para cada $a$: marca dois $b$'s e um $c$ (exceto o primeiro $a$ que não marca $c$)
    \item Verifica que a contagem está correta
\end{enumerate}

\subsubsection*{Diagrama de Estados}
\begin{figure}[H]
    \centering
    \includegraphics[width=0.8\textwidth,height=0.4\textheight,keepaspectratio]{diagramas/MT_exe1_g.pdf}
    \caption{MT para $L = \{a^n b^{2n} c^{n-1}\}$}
\end{figure}

\newpage
%------------------------------------------------------------------------------
\subsection{1.h) $L = \{a^i b^j a^k \mid j = \max(i, k)\}$}
%------------------------------------------------------------------------------

\subsubsection*{Descrição}
Esta MT aceita cadeias onde o número de $b$'s é igual ao máximo entre o número de $a$'s à esquerda e à direita.

\subsubsection*{Algoritmo}
\begin{enumerate}
    \item Pareia $a$'s da esquerda com $a$'s da direita
    \item O excedente (se houver) deve ser igual aos $b$'s
    \item Os $b$'s pareados com $a$'s dos dois lados devem bater
\end{enumerate}

\subsubsection*{Diagrama de Estados}
\begin{figure}[H]
    \centering
    \includegraphics[width=0.8\textwidth,height=0.4\textheight,keepaspectratio]{diagramas/MT_exe1_h.pdf}
    \caption{MT para $L = \{a^i b^j a^k \mid j = \max(i, k)\}$}
\end{figure}

\newpage
%------------------------------------------------------------------------------
\subsection{1.i) $L = \{a^i b^j a^k \mid i = j \text{ ou } j = k\}$}
%------------------------------------------------------------------------------

\subsubsection*{Descrição}
Esta MT aceita cadeias onde o número de $b$'s é igual ao número de $a$'s à esquerda \textbf{OU} ao número de $a$'s à direita.

\subsubsection*{Algoritmo}
A MT verifica ambas as condições:
\begin{enumerate}
    \item \textbf{Caso $i = j$:} Pareia cada $a$ da esquerda com um $b$
    \item \textbf{Caso $j = k$:} Pareia cada $b$ com um $a$ da direita
    \item Aceita se qualquer uma das condições for satisfeita
\end{enumerate}

\subsubsection*{Complexidade}
\begin{itemize}
    \item \textbf{Estados:} 27
    \item \textbf{Transições:} 116
\end{itemize}

\subsubsection*{Diagrama de Estados}
\begin{figure}[H]
    \centering
    \includegraphics[width=0.8\textwidth,height=0.4\textheight,keepaspectratio]{diagramas/MT_exe1_i.pdf}
    \caption{MT para $L = \{a^i b^j a^k \mid i = j \text{ ou } j = k\}$}
\end{figure}

\subsubsection*{Versão sem Estado de Rejeição Explícito}
\begin{figure}[H]
    \centering
    \includegraphics[width=0.8\textwidth,height=0.4\textheight,keepaspectratio]{diagramas/MT_exe1_i_sem_qr.pdf}
    \caption{MT para $L = \{a^i b^j a^k \mid i = j \text{ ou } j = k\}$ (sem transições para $q_r$)}
\end{figure}

%%%%%%%%%%%%%%%%%%%%%%%%%%%%%%%%%%%%%%%%%%%%%%%%%%%%%%%%%%%%%%%%%%%%%%%%%%%%%%%
\newpage
\section{Exercício 2: Análise de Máquina de Turing}
%%%%%%%%%%%%%%%%%%%%%%%%%%%%%%%%%%%%%%%%%%%%%%%%%%%%%%%%%%%%%%%%%%%%%%%%%%%%%%%

\textbf{Enunciado:} Dada a MT da figura abaixo, sobre o alfabeto $\Sigma = \{a, b\}$, responda:

\textit{O símbolo \# não pertence ao alfabeto e foi inserido somente para marcar o início da fita.}

%------------------------------------------------------------------------------
\subsection{Diagrama de Estados da MT}
%------------------------------------------------------------------------------

\begin{figure}[H]
    \centering
    \includegraphics[width=0.8\textwidth,height=0.4\textheight,keepaspectratio]{diagramas/MT_exe2.pdf}
    \caption{Máquina de Turing para $L = \{w \mid n_a(w) = n_b(w)\}$}
\end{figure}

%------------------------------------------------------------------------------
\subsection{Definição Formal da MT}
%------------------------------------------------------------------------------

$$M = (Q, \Sigma, \Gamma, \delta, q_0, q_{aceita}, q_{rejeita})$$

\begin{table}[h]
\centering
\begin{tabular}{|l|l|}
\hline
\textbf{Componente} & \textbf{Valor} \\
\hline
Estados ($Q$) & $\{q_0, q_1, q_2, q_3, q_4, q_5, q_r\}$ \\
Alfabeto de Entrada ($\Sigma$) & $\{a, b\}$ \\
Alfabeto da Fita ($\Gamma$) & $\{a, b, \#, X, \_\}$ \\
Símbolo Branco & $\_$ \\
Estado Inicial & $q_0$ \\
Estado de Aceitação & $q_5$ \\
Estado de Rejeição & $q_r$ (implícito) \\
\hline
\end{tabular}
\caption{Componentes da Máquina de Turing}
\end{table}

\subsubsection*{Notação Formal das Transições}
A função de transição $\delta: Q \times \Gamma \rightarrow Q \times \Gamma \times \{L, R, N\}$ é definida pelas seguintes regras:

\begin{align}
\delta(q_0, \#) &= (q_1, \#, R) \\
\delta(q_0, \_) &= (q_5, \_, N) \\
\delta(q_1, a) &= (q_2, X, R) \\
\delta(q_1, b) &= (q_4, X, R) \\
\delta(q_1, X) &= (q_1, X, R) \\
\delta(q_1, \_) &= (q_5, \_, N) \\
\delta(q_2, a) &= (q_2, a, R) \\
\delta(q_2, X) &= (q_2, X, R) \\
\delta(q_2, b) &= (q_3, X, L) \\
\delta(q_3, a) &= (q_3, a, L) \\
\delta(q_3, b) &= (q_3, b, L) \\
\delta(q_3, X) &= (q_3, X, L) \\
\delta(q_3, \#) &= (q_1, \#, R) \\
\delta(q_4, b) &= (q_4, b, R) \\
\delta(q_4, X) &= (q_4, X, R) \\
\delta(q_4, a) &= (q_3, X, L)
\end{align}

%------------------------------------------------------------------------------
\subsection{Função de Transição ($\delta$)}
%------------------------------------------------------------------------------

\begin{table}[h]
\centering
\small
\begin{tabular}{|c|c|c|c|c|p{5cm}|}
\hline
\textbf{Estado} & \textbf{Lê} & \textbf{Próximo} & \textbf{Escreve} & \textbf{Move} & \textbf{Descrição} \\
\hline
$q_0$ & $\#$ & $q_1$ & $\#$ & R & Início: Move para direita \\
$q_0$ & $\_$ & $q_5$ & $\_$ & N & String vazia: aceita \\
\hline
$q_1$ & $a$ & $q_2$ & $X$ & R & Marca $a$ e busca $b$ \\
$q_1$ & $b$ & $q_4$ & $X$ & R & Marca $b$ e busca $a$ \\
$q_1$ & $X$ & $q_1$ & $X$ & R & Pula marcados \\
$q_1$ & $\_$ & $q_5$ & $\_$ & N & Todos pareados: aceita \\
\hline
$q_2$ & $a$ & $q_2$ & $a$ & R & Pula $a$'s \\
$q_2$ & $X$ & $q_2$ & $X$ & R & Pula $X$'s \\
$q_2$ & $b$ & $q_3$ & $X$ & L & Encontrou $b$: marca e retorna \\
\hline
$q_3$ & $a$ & $q_3$ & $a$ & L & Retorna pulando $a$ \\
$q_3$ & $b$ & $q_3$ & $b$ & L & Retorna pulando $b$ \\
$q_3$ & $X$ & $q_3$ & $X$ & L & Retorna pulando $X$ \\
$q_3$ & $\#$ & $q_1$ & $\#$ & R & Chegou ao início \\
\hline
$q_4$ & $b$ & $q_4$ & $b$ & R & Pula $b$'s \\
$q_4$ & $X$ & $q_4$ & $X$ & R & Pula $X$'s \\
$q_4$ & $a$ & $q_3$ & $X$ & L & Encontrou $a$: marca e retorna \\
\hline
\end{tabular}
\caption{Função de Transição (16 transições)}
\end{table}

%------------------------------------------------------------------------------
\newpage
\subsection{Questão a) Sequência de Configurações}
%------------------------------------------------------------------------------

\subsubsection*{Cadeia: \texttt{\#abab}}

\begin{acceptbox}
\textbf{Resultado: ACEITA} (2 $a$'s e 2 $b$'s pareados)
\end{acceptbox}

\begin{verbatim}
Passo 0:  q0, [#]abab    → (q0, #) -> (q1, #, R)
Passo 1:  q1, #[a]bab    → (q1, a) -> (q2, X, R)
Passo 2:  q2, #X[b]ab    → (q2, b) -> (q3, X, L)
Passo 3:  q3, #[X]Xab    → (q3, X) -> (q3, X, L)
Passo 4:  q3, [#]XXab    → (q3, #) -> (q1, #, R)
Passo 5:  q1, #[X]Xab    → (q1, X) -> (q1, X, R)
Passo 6:  q1, #X[X]ab    → (q1, X) -> (q1, X, R)
Passo 7:  q1, #XX[a]b    → (q1, a) -> (q2, X, R)
Passo 8:  q2, #XXX[b]    → (q2, b) -> (q3, X, L)
  ...     (retorno para #)
Passo 17: q1, #XXXX[_]   → (q1, _) -> (q5, _, N) ✓
\end{verbatim}

\subsubsection*{Cadeia: \texttt{\#aabba}}

\begin{rejectbox}
\textbf{Resultado: REJEITA} (3 $a$'s e 2 $b$'s - sobrou 1 $a$ sem par)
\end{rejectbox}

\begin{verbatim}
Passo 0:  q0, [#]aabba   → (q0, #) -> (q1, #, R)
Passo 1:  q1, #[a]abba   → (q1, a) -> (q2, X, R)
Passo 2:  q2, #X[a]bba   → (q2, a) -> (q2, a, R)
Passo 3:  q2, #Xa[b]ba   → (q2, b) -> (q3, X, L)
  ...     (pareamento de a-b, a-b)
Passo 19: q1, #XXXX[a]   → (q1, a) -> (q2, X, R)
Passo 20: q2, #XXXXX[_]  → SEM TRANSIÇÃO! ✗
\end{verbatim}

\subsubsection*{Cadeia: \texttt{\#bbabaa}}

\begin{acceptbox}
\textbf{Resultado: ACEITA} (3 $a$'s e 3 $b$'s pareados)
\end{acceptbox}

\begin{verbatim}
Passo 0:  q0, [#]bbabaa  → (q0, #) -> (q1, #, R)
Passo 1:  q1, #[b]babaa  → (q1, b) -> (q4, X, R)
Passo 2:  q4, #X[b]abaa  → (q4, b) -> (q4, b, R)
Passo 3:  q4, #Xb[a]baa  → (q4, a) -> (q3, X, L)
  ...     (pareamento de b-a, b-a, b-a)
Passo 35: q1, #XXXXXX[_] → (q1, _) -> (q5, _, N) ✓
\end{verbatim}

\subsubsection*{Cadeia: \texttt{\#abaabab}}

\begin{rejectbox}
\textbf{Resultado: REJEITA} (4 $a$'s e 3 $b$'s - sobrou 1 $a$ sem par)
\end{rejectbox}

\begin{verbatim}
Passo 0:  q0, [#]abaabab → (q0, #) -> (q1, #, R)
Passo 1:  q1, #[a]baabab → (q1, a) -> (q2, X, R)
Passo 2:  q2, #X[b]aabab → (q2, b) -> (q3, X, L)
  ...     (pareamento de a-b, a-b, a-b)
Passo 34: q1, #XXXXX[a]X → (q1, a) -> (q2, X, R)
Passo 35: q2, #XXXXXX[X] → (q2, X) -> (q2, X, R)
Passo 36: q2, #XXXXXXX[_] → SEM TRANSIÇÃO! ✗
\end{verbatim}

%------------------------------------------------------------------------------
\subsection{Resumo da Questão a)}
%------------------------------------------------------------------------------

\begin{table}[h]
\centering
\begin{tabular}{|c|c|c|c|}
\hline
\textbf{Cadeia} & \textbf{\# de $a$'s} & \textbf{\# de $b$'s} & \textbf{Resultado} \\
\hline
\texttt{\#abab} & 2 & 2 & \textcolor{green!50!black}{\textbf{ACEITA}} \\
\texttt{\#aabba} & 3 & 2 & \textcolor{red!50!black}{\textbf{REJEITA}} \\
\texttt{\#bbabaa} & 3 & 3 & \textcolor{green!50!black}{\textbf{ACEITA}} \\
\texttt{\#abaabab} & 4 & 3 & \textcolor{red!50!black}{\textbf{REJEITA}} \\
\hline
\end{tabular}
\caption{Resultados para as cadeias da questão a)}
\end{table}

%------------------------------------------------------------------------------
\newpage
\subsection{Questão b) Linguagem Aceita}
%------------------------------------------------------------------------------

\begin{tcolorbox}[colback=blue!5,colframe=blue!75!black,title=Resposta]
A linguagem $L$ aceita por essa Máquina de Turing é:

$$\boxed{L = \{ w \in \{a, b\}^* \mid n_a(w) = n_b(w) \}}$$

Onde:
\begin{itemize}
    \item $n_a(w)$ = número de ocorrências de $a$ na cadeia $w$
    \item $n_b(w)$ = número de ocorrências de $b$ na cadeia $w$
\end{itemize}
\end{tcolorbox}

\subsubsection*{Descrição}
A MT aceita todas as cadeias sobre $\{a, b\}$ que contêm \textbf{exatamente o mesmo número de $a$'s e $b$'s}, independente da ordem em que aparecem.

\subsubsection*{Exemplos de cadeias aceitas}
\begin{itemize}
    \item $\epsilon$ (cadeia vazia)
    \item $ab$, $ba$
    \item $aabb$, $abab$, $abba$, $baab$, $baba$, $bbaa$
    \item $aaabbb$, $ababab$, $aababb$, etc.
\end{itemize}

\subsubsection*{Exemplos de cadeias rejeitadas}
\begin{itemize}
    \item $a$, $b$
    \item $aab$, $abb$, $aaa$, $bbb$
    \item $aabba$, $abbba$, etc.
\end{itemize}

\subsubsection*{Funcionamento do Algoritmo}
\begin{enumerate}
    \item \textbf{Início ($q_0$):} Lê o marcador \# e vai para $q_1$
    \item \textbf{Busca de par ($q_1$):}
    \begin{itemize}
        \item Se encontra $a$: marca com $X$, vai para $q_2$ (buscar $b$)
        \item Se encontra $b$: marca com $X$, vai para $q_4$ (buscar $a$)
        \item Se encontra $\_$: todos pareados $\rightarrow$ \textbf{ACEITA}
    \end{itemize}
    \item \textbf{Busca de $b$ ($q_2$):} Avança até encontrar $b$, marca e retorna
    \item \textbf{Busca de $a$ ($q_4$):} Avança até encontrar $a$, marca e retorna
    \item \textbf{Retorno ($q_3$):} Volta ao início para próximo par
    \item \textbf{Aceitação ($q_5$):} Estado final
\end{enumerate}

%%%%%%%%%%%%%%%%%%%%%%%%%%%%%%%%%%%%%%%%%%%%%%%%%%%%%%%%%%%%%%%%%%%%%%%%%%%%%%%
\newpage
\section{Exercício 3: MT Multifita}
%%%%%%%%%%%%%%%%%%%%%%%%%%%%%%%%%%%%%%%%%%%%%%%%%%%%%%%%%%%%%%%%%%%%%%%%%%%%%%%

\textbf{Enunciado:} Refaça os exercícios usando MT multifita com uma complexidade de tempo inferior à MT padrão.

\subsection{Introdução às MT Multifita}

Uma \textbf{Máquina de Turing Multifita} possui $k$ fitas, cada uma com seu próprio cabeçote de leitura/escrita. Em cada passo, a máquina:
\begin{enumerate}
    \item Lê os símbolos sob todos os $k$ cabeçotes simultaneamente
    \item Dependendo do estado e dos símbolos lidos, transita para um novo estado
    \item Escreve novos símbolos em cada fita e move cada cabeçote independentemente
\end{enumerate}

\subsubsection*{Vantagem de Complexidade}
\begin{tcolorbox}[colback=blue!5,colframe=blue!75!black,title=Teorema]
Uma MT multifita com $k$ fitas pode simular qualquer computação que uma MT padrão faz em tempo $T(n)$ em tempo $O(T(n)^2)$. Porém, para muitos problemas, a MT multifita pode resolver em $O(n)$ o que a MT padrão resolve em $O(n^2)$.
\end{tcolorbox}

A principal vantagem é usar fitas extras como \textbf{contadores} ou \textbf{buffers}, evitando múltiplas passagens pela entrada.

%------------------------------------------------------------------------------
\newpage
\subsection{3.a) $L = \{w \mid w \text{ começa com } ab\}$}
%------------------------------------------------------------------------------

\begin{tcolorbox}[colback=gray!10,colframe=gray!50,title=Análise de Complexidade]
\textbf{MT Padrão:} $O(1)$ - Apenas verifica 2 símbolos\\
\textbf{MT Multifita:} Não há ganho - O problema já é trivial
\end{tcolorbox}

Este problema não se beneficia de múltiplas fitas pois a verificação é feita em tempo constante.

%------------------------------------------------------------------------------
\newpage
\subsection{3.b) $L = \{a^n b^n c^n \mid n \geq 0\}$}
%------------------------------------------------------------------------------

\subsubsection*{Análise de Complexidade}
\begin{tcolorbox}[colback=gray!10,colframe=gray!50,title=Comparação]
\textbf{MT Padrão:} $O(n^2)$ - Para cada grupo de $(a,b,c)$, percorre a fita inteira\\
\textbf{MT Multifita (3 fitas):} $O(n)$ - Uma única passagem
\end{tcolorbox}

\subsubsection*{Algoritmo Multifita}
\begin{enumerate}
    \item \textbf{Fita 1:} Entrada original
    \item \textbf{Fita 2:} Contador de $a$'s (representação unária)
    \item \textbf{Fita 3:} Contador de $b$'s
\end{enumerate}

\begin{verbatim}
Passo 1: Percorre a's na fita 1, escrevendo I na fita 2 para cada a
Passo 2: Percorre b's na fita 1:
         - Para cada b, remove um I da fita 2
         - Adiciona um I na fita 3
Passo 3: Percorre c's na fita 1:
         - Para cada c, remove um I da fita 3
Passo 4: Se fita 2 e fita 3 estão vazias → ACEITA
\end{verbatim}

\subsubsection*{Transições (notação: [fita1, fita2, fita3])}
\begin{align*}
(q_0, [a, \_, \_]) &\rightarrow (q_0, [a, I, \_], [R, R, N]) \\
(q_0, [b, I, \_]) &\rightarrow (q_1, [b, \_, I], [R, L, R]) \\
(q_1, [b, I, \_]) &\rightarrow (q_1, [b, \_, I], [R, L, R]) \\
(q_1, [c, \_, I]) &\rightarrow (q_2, [c, \_, \_], [R, N, L]) \\
(q_2, [c, \_, I]) &\rightarrow (q_2, [c, \_, \_], [R, N, L]) \\
(q_2, [\_, \_, \_]) &\rightarrow (q_f, [\_, \_, \_], [N, N, N])
\end{align*}

%------------------------------------------------------------------------------
\newpage
\subsection{3.c) $L = \{a^n b^m \mid n = 2m\}$}
%------------------------------------------------------------------------------

\subsubsection*{Análise de Complexidade}
\begin{tcolorbox}[colback=gray!10,colframe=gray!50,title=Comparação]
\textbf{MT Padrão:} $O(n^2)$ - Marca 2 $a$'s e 1 $b$ por passagem\\
\textbf{MT Multifita (2 fitas):} $O(n)$ - Conta $a$'s, desconta 2 por $b$
\end{tcolorbox}

\subsubsection*{Algoritmo Multifita}
\begin{enumerate}
    \item \textbf{Fita 1:} Entrada
    \item \textbf{Fita 2:} Contador de $a$'s
\end{enumerate}

\begin{verbatim}
Passo 1: Para cada 'a' na fita 1, escreve 'I' na fita 2
Passo 2: Para cada 'b' na fita 1, apaga DOIS 'I' da fita 2
Passo 3: Se fita 2 vazia ao final → ACEITA
\end{verbatim}

\subsubsection*{Exemplo: $aaaaabb$ (aceita: $n=4$, $m=2$, $4=2\times2$)}
\begin{verbatim}
Fita 1: [a]aaaabb    Fita 2: [_]
Fita 1: a[a]aabb     Fita 2: I[_]
Fita 1: aa[a]abb     Fita 2: II[_]
Fita 1: aaa[a]bb     Fita 2: III[_]
Fita 1: aaaa[b]b     Fita 2: IIII[_] → apaga 2: II[_]
Fita 1: aaaab[b]     Fita 2: II[_] → apaga 2: [_]
Fita 1: aaaaabb[_]   Fita 2: [_] → ACEITA ✓
\end{verbatim}

%------------------------------------------------------------------------------
\newpage
\subsection{3.d) $L = \{ww^R \mid w \in \{a, b\}^*\}$ (Palíndromos pares)}
%------------------------------------------------------------------------------

\subsubsection*{Análise de Complexidade}
\begin{tcolorbox}[colback=gray!10,colframe=gray!50,title=Comparação]
\textbf{MT Padrão:} $O(n^2)$ - Compara primeiro com último, volta ao início\\
\textbf{MT Multifita (2 fitas):} $O(n)$ - Copia e compara em paralelo
\end{tcolorbox}

\subsubsection*{Algoritmo Multifita}
\begin{enumerate}
    \item \textbf{Fita 1:} Entrada (lida da esquerda para direita)
    \item \textbf{Fita 2:} Cópia da entrada (lida da direita para esquerda)
\end{enumerate}

\begin{verbatim}
Passo 1: Copia toda a entrada da fita 1 para a fita 2
Passo 2: Move cabeçote da fita 1 para o início
Passo 3: Move cabeçote da fita 2 para o fim
Passo 4: Compara símbolo por símbolo:
         - Fita 1 avança (→), Fita 2 retrocede (←)
         - Se todos iguais → ACEITA
\end{verbatim}

\subsubsection*{Exemplo: $abba$ (aceita: $w=ab$, $w^R=ba$)}
\begin{verbatim}
Após cópia:
Fita 1: [a]bba_    (leitura →)
Fita 2: _abba[_]   (leitura ←, posiciona no fim)

Comparação:
Fita 1: [a]bba    Fita 2: abb[a] → a=a ✓
Fita 1: a[b]ba    Fita 2: ab[b]a → b=b ✓
Fita 1: ab[b]a    Fita 2: a[b]ba → b=b ✓
Fita 1: abb[a]    Fita 2: [a]bba → a=a ✓
ACEITA!
\end{verbatim}

%------------------------------------------------------------------------------
\newpage
\subsection{3.e) $L = \{w \mid n_a(w) = n_b(w)\}$}
%------------------------------------------------------------------------------

\subsubsection*{Análise de Complexidade}
\begin{tcolorbox}[colback=gray!10,colframe=gray!50,title=Comparação]
\textbf{MT Padrão:} $O(n^2)$ - Pareia cada $a$ com um $b$, volta ao início\\
\textbf{MT Multifita (2 fitas):} $O(n)$ - Usa contador na fita 2
\end{tcolorbox}

\subsubsection*{Algoritmo Multifita}
\begin{enumerate}
    \item \textbf{Fita 1:} Entrada
    \item \textbf{Fita 2:} Contador (diferença entre $a$'s e $b$'s)
\end{enumerate}

\begin{verbatim}
Para cada símbolo na fita 1:
  - Se 'a': incrementa contador (escreve I na fita 2)
  - Se 'b': decrementa contador (apaga I da fita 2)
    - Se contador já vazio, marca como negativo
No final: aceita se contador = 0
\end{verbatim}

\subsubsection*{Exemplo: $abba$ (aceita: 2 $a$'s e 2 $b$'s)}
\begin{verbatim}
Fita 1: [a]bba    Fita 2: [_] → escreve I → [I]
Fita 1: a[b]ba    Fita 2: [I] → apaga I → [_]
Fita 1: ab[b]a    Fita 2: [_] → marca negativo → [-]
Fita 1: abb[a]    Fita 2: [-] → cancela negativo → [_]
Fita 1: abba[_]   Fita 2: [_] → contador = 0 → ACEITA ✓
\end{verbatim}

%------------------------------------------------------------------------------
\newpage
\subsection{3.f) $L = \{1^n 0^{n+3} \mid n \geq 0\}$}
%------------------------------------------------------------------------------

\subsubsection*{Análise de Complexidade}
\begin{tcolorbox}[colback=gray!10,colframe=gray!50,title=Comparação]
\textbf{MT Padrão:} $O(n^2)$ - Marca 1's e 0's em múltiplas passagens\\
\textbf{MT Multifita (2 fitas):} $O(n)$ - Conta 1's, pula 3 zeros, compara
\end{tcolorbox}

\subsubsection*{Algoritmo Multifita}
\begin{verbatim}
Passo 1: Para cada '1' na fita 1, escreve 'I' na fita 2
Passo 2: Lê exatamente 3 zeros (os "+3" obrigatórios)
Passo 3: Para cada '0' restante, apaga um 'I' da fita 2
Passo 4: Se fita 2 vazia → ACEITA
\end{verbatim}

\subsubsection*{Exemplo: $110000$ (aceita: $n=2$, $2+3=5$... ops, rejeitaria)}
\subsubsection*{Exemplo: $11000000$ (aceita: $n=2$, $0$'s$=5=2+3$)}
\begin{verbatim}
Fita 1: [1]1000000   Fita 2: [_] → I
Fita 1: 1[1]000000   Fita 2: I[_] → II
Fita 1: 11[0]00000   Fita 2: II (pula 3 zeros)
Fita 1: 110[0]0000   Fita 2: II
Fita 1: 1100[0]000   Fita 2: II
Fita 1: 11000[0]00   Fita 2: I[I] → apaga → I
Fita 1: 110000[0]0   Fita 2: [I] → apaga → _
Fita 1: 1100000[_]   Fita 2: [_] → ACEITA ✓
\end{verbatim}

%------------------------------------------------------------------------------
\newpage
\subsection{3.g) $L = \{a^n b^{2n} c^{n-1} \mid n > 0\}$}
%------------------------------------------------------------------------------

\subsubsection*{Análise de Complexidade}
\begin{tcolorbox}[colback=gray!10,colframe=gray!50,title=Comparação]
\textbf{MT Padrão:} $O(n^2)$ - Múltiplas passagens marcando símbolos\\
\textbf{MT Multifita (2 fitas):} $O(n)$ - Conta e verifica em uma passagem
\end{tcolorbox}

\subsubsection*{Algoritmo Multifita}
\begin{verbatim}
Passo 1: Para cada 'a', escreve 'I' na fita 2 (conta n)
Passo 2: Para cada 'b', apaga 0.5 'I' (ou seja, 2 b's = 1 I)
         Alternativa: para cada par de b's, apaga 1 'I'
Passo 3: Ao acabar b's, fita 2 deve estar vazia (confirma 2n)
Passo 4: Reconta os a's: n-1 deve ser igual ao número de c's
         - Pula primeiro c (é o "-1")
         - Para cada c seguinte, verifica correspondência
\end{verbatim}

\subsubsection*{Exemplo: $abbbc$ (aceita: $n=2$, $b=4=2\times2$, $c=1=2-1$)}
\begin{verbatim}
Fita 1: [a]abbbc    Fita 2: [_] → I
Fita 1: a[a]bbbc    Fita 2: I[_] → II
Fita 1: aa[b]bbc    Fita 2: II → par de b's → I
Fita 1: aab[b]bc    (continua par)
Fita 1: aabb[b]c    Fita 2: I → par de b's → _
Fita 1: aabbb[b]    (continua par)
Fita 1: aabbbb[c]   Fita 2: [_], primeiro c (grátis)
Fita 1: aabbbbc[_]  → ACEITA ✓
\end{verbatim}

%------------------------------------------------------------------------------
\newpage
\subsection{3.h) $L = \{a^i b^j a^k \mid j = \max(i, k)\}$}
%------------------------------------------------------------------------------

\subsubsection*{Análise de Complexidade}
\begin{tcolorbox}[colback=gray!10,colframe=gray!50,title=Comparação]
\textbf{MT Padrão:} $O(n^2)$ - Múltiplas comparações\\
\textbf{MT Multifita (3 fitas):} $O(n)$ - Conta i, j, k e compara
\end{tcolorbox}

\subsubsection*{Algoritmo Multifita}
\begin{enumerate}
    \item \textbf{Fita 1:} Entrada
    \item \textbf{Fita 2:} Contador de $i$ (a's à esquerda)
    \item \textbf{Fita 3:} Contador de $k$ (a's à direita)
\end{enumerate}

\begin{verbatim}
Passo 1: Conta a's à esquerda na fita 2
Passo 2: Conta b's (guarda contagem j)
Passo 3: Conta a's à direita na fita 3
Passo 4: Determina max(i, k):
         - Compara fitas 2 e 3 símbolo a símbolo
         - O maior é max(i, k)
Passo 5: Verifica se j = max(i, k)
\end{verbatim}

\subsubsection*{Exemplo: $aabbba$ (aceita: $i=2$, $j=3$, $k=1$, $\max(2,1)=2$... rejeita)}
\subsubsection*{Exemplo: $aabba$ (aceita: $i=2$, $j=2$, $k=1$, $\max(2,1)=2=j$)}
\begin{verbatim}
Fita 1: [a]abba     Fita 2: [_]→I  Fita 3: [_]
Fita 1: a[a]bba     Fita 2: I[_]→II
Fita 1: aa[b]ba     (conta j=1)
Fita 1: aab[b]a     (conta j=2)
Fita 1: aabb[a]     Fita 3: [_]→I
Fita 1: aabba[_]    
Comparação: Fita 2 = II (i=2), Fita 3 = I (k=1)
max(2,1) = 2 = j → ACEITA ✓
\end{verbatim}

%------------------------------------------------------------------------------
\newpage
\subsection{3.i) $L = \{a^i b^j a^k \mid i = j \text{ ou } j = k\}$}
%------------------------------------------------------------------------------

\subsubsection*{Análise de Complexidade}
\begin{tcolorbox}[colback=gray!10,colframe=gray!50,title=Comparação]
\textbf{MT Padrão:} $O(n^2)$ ou $O(n^3)$ - Tenta ambas condições\\
\textbf{MT Multifita (3 fitas):} $O(n)$ - Conta tudo e verifica condições
\end{tcolorbox}

\subsubsection*{Algoritmo Multifita}
\begin{enumerate}
    \item \textbf{Fita 1:} Entrada
    \item \textbf{Fita 2:} Contador de $i$
    \item \textbf{Fita 3:} Contador de $k$
\end{enumerate}

\begin{verbatim}
Passo 1: Conta a's à esquerda na fita 2 (i)
Passo 2: Conta b's, e simultaneamente:
         - Apaga um I da fita 2 para cada b
         - Se fita 2 esvazia com os b's → i = j ✓
Passo 3: Conta a's à direita na fita 3 (k)
Passo 4: Compara j com k:
         - Se fita 3 tem exatamente j I's → j = k ✓
Passo 5: ACEITA se qualquer condição for verdadeira
\end{verbatim}

\subsubsection*{Exemplo: $aabba$ ($i=2$, $j=2$, $k=1$, $i=j$ ✓)}
\begin{verbatim}
Fita 1: [a]abba     Fita 2: [_]→I
Fita 1: a[a]bba     Fita 2: I[_]→II
Fita 1: aa[b]ba     Fita 2: I[I]→I (apaga 1 para b)
Fita 1: aab[b]a     Fita 2: [I]→_ (apaga 1 para b)
Fita 2 vazia! → i = j confirmado → ACEITA ✓
\end{verbatim}

\subsubsection*{Exemplo: $abba$ ($i=1$, $j=2$, $k=1$, $j=k$? Não, $2\neq1$. $i=j$? Não.)}
Rejeita.

\subsubsection*{Exemplo: $abbaa$ ($i=1$, $j=2$, $k=2$, $j=k$ ✓)}
\begin{verbatim}
Fita 1: [a]bbaa     Fita 2: I
Fita 1: a[b]baa     Fita 2: _ (i≠j, continua)
Fita 1: ab[b]aa     Contador j = 2
Fita 1: abb[a]a     Fita 3: I
Fita 1: abba[a]     Fita 3: II
Comparação: j=2, Fita 3=II (k=2) → j = k → ACEITA ✓
\end{verbatim}

%------------------------------------------------------------------------------
\newpage
\subsection{Resumo: Comparação de Complexidades}
%------------------------------------------------------------------------------

\begin{table}[H]
\centering
\begin{tabular}{|c|l|c|c|c|}
\hline
\textbf{\#} & \textbf{Linguagem} & \textbf{MT Padrão} & \textbf{MT Multifita} & \textbf{Fitas} \\
\hline
a & $w$ começa com $ab$ & $O(1)$ & $O(1)$ & 1 \\
b & $a^n b^n c^n$ & $O(n^2)$ & $O(n)$ & 3 \\
c & $a^n b^m, n=2m$ & $O(n^2)$ & $O(n)$ & 2 \\
d & $ww^R$ & $O(n^2)$ & $O(n)$ & 2 \\
e & $n_a = n_b$ & $O(n^2)$ & $O(n)$ & 2 \\
f & $1^n 0^{n+3}$ & $O(n^2)$ & $O(n)$ & 2 \\
g & $a^n b^{2n} c^{n-1}$ & $O(n^2)$ & $O(n)$ & 2 \\
h & $j = \max(i,k)$ & $O(n^2)$ & $O(n)$ & 3 \\
i & $i=j$ ou $j=k$ & $O(n^2)$ & $O(n)$ & 3 \\
\hline
\end{tabular}
\caption{Comparação de complexidades entre MT padrão e MT multifita}
\end{table}

\begin{tcolorbox}[colback=green!5,colframe=green!50!black,title=Conclusão]
A MT multifita reduz a complexidade de $O(n^2)$ para $O(n)$ na maioria dos problemas, utilizando fitas extras como \textbf{contadores} ou \textbf{buffers de comparação}. O princípio geral é:
\begin{itemize}
    \item Evitar múltiplas passagens pela entrada
    \item Usar fitas auxiliares para armazenar contagens
    \item Processar a entrada em uma única varredura
\end{itemize}
\end{tcolorbox}

%%%%%%%%%%%%%%%%%%%%%%%%%%%%%%%%%%%%%%%%%%%%%%%%%%%%%%%%%%%%%%%%%%%%%%%%%%%%%%%
\newpage
\section{Exercício 4: MT Não-Determinística}
%%%%%%%%%%%%%%%%%%%%%%%%%%%%%%%%%%%%%%%%%%%%%%%%%%%%%%%%%%%%%%%%%%%%%%%%%%%%%%%

\textbf{Enunciado:} Refaça os exercícios usando MT Não-Determinísticas com uma complexidade de tempo inferior à MT padrão.

\vspace{1cm}
\textit{A ser implementado...}

%%%%%%%%%%%%%%%%%%%%%%%%%%%%%%%%%%%%%%%%%%%%%%%%%%%%%%%%%%%%%%%%%%%%%%%%%%%%%%%
\newpage
\section*{Anexo: Comandos CLI para Testes}
%%%%%%%%%%%%%%%%%%%%%%%%%%%%%%%%%%%%%%%%%%%%%%%%%%%%%%%%%%%%%%%%%%%%%%%%%%%%%%%

Os arquivos JSON das Máquinas de Turing podem ser testados usando o CLI:

\begin{lstlisting}[language=bash,basicstyle=\ttfamily\small]
# Testar exercicio 1.a
node cli.js --def input/MT_exe1_a.json --test "ab,abc,abcc,ba,a"

# Testar exercicio 1.i
node cli.js --def input/MT_exe1_i.json --test "ab,aab,abb,aabb" --verbose

# Testar exercicio 2 (cadeias da questao a)
node cli.js --def input/MT_exe2.json --test "#abab,#aabba,#bbabaa,#abaabab"

# Teste com saida detalhada
node cli.js --def input/MT_exe2.json --test "#abab" --verbose
\end{lstlisting}

\end{document}
