% !TEX program = pdflatex
\documentclass[a4paper,12pt]{article}

% ================== PACOTES ==================
\usepackage[utf8]{inputenc}
\usepackage[T1]{fontenc}
\usepackage[brazilian]{babel}
\usepackage{amsmath,amssymb,amsthm}
\usepackage{graphicx}
\usepackage{float}
\usepackage{hyperref}
\usepackage{fancyhdr}
\usepackage{geometry}
\usepackage{tcolorbox}
\usepackage{enumitem}
\usepackage{booktabs}
\usepackage{multirow}
\usepackage{array}
\usepackage{xcolor}
\usepackage{tikz}

% ================== CONFIGURAÇÕES ==================
\geometry{
    left=2.5cm,
    right=2.5cm,
    top=2.5cm,
    bottom=2.5cm
}

\hypersetup{
    colorlinks=true,
    linkcolor=blue,
    filecolor=magenta,
    urlcolor=cyan,
}

% ================== CABEÇALHO/RODAPÉ ==================
\pagestyle{fancy}
\fancyhf{}
\fancyhead[L]{LFA - Linguagens Formais e Autômatos}
\fancyhead[R]{CEFET-MG}
\fancyfoot[C]{\thepage}
\renewcommand{\headrulewidth}{0.4pt}
\renewcommand{\footrulewidth}{0.4pt}

% ================== BOXES ==================
\newtcolorbox{erbox}{
    colback=blue!5,
    colframe=blue!60!black,
    title={\textbf{Expressão Regular}},
    fonttitle=\bfseries
}

\newtcolorbox{examplebox}{
    colback=green!5,
    colframe=green!60!black,
    title={\textbf{Exemplos}},
    fonttitle=\bfseries
}

\newtcolorbox{explanationbox}{
    colback=yellow!5,
    colframe=orange!60!black,
    title={\textbf{Explicação}},
    fonttitle=\bfseries
}

\newtcolorbox{acceptbox}{
    colback=green!10,
    colframe=green!60!black,
    title={\textbf{Aceita}},
    fonttitle=\bfseries
}

\newtcolorbox{rejectbox}{
    colback=red!10,
    colframe=red!60!black,
    title={\textbf{Rejeita}},
    fonttitle=\bfseries
}

% ================== TÍTULO ==================
\title{\textbf{Lista de Exercícios}\\
\Large Expressões Regulares (ER)\\
\large Resoluções Comentadas}
\author{Linguagens Formais e Autômatos}
\date{\today}

\begin{document}

\maketitle
\tableofcontents
\newpage

% ============================================================
% EXERCÍCIO 1
% ============================================================
\section{Exercício 1: Linguagens sobre \{0,1\}}

\textbf{Descreva mais formalmente as seguintes linguagens sobre o alfabeto $\Sigma = \{0,1\}$:}

\subsection{(a) Palavras com, no mínimo, um 0}

\textbf{Descrição:} O conjunto das palavras com, no mínimo, um 0.

\begin{erbox}
\textbf{Expressão Regular:}
$$(0|1)^* \cdot 0 \cdot (0|1)^*$$

Ou de forma equivalente:
$$\Sigma^* \cdot 0 \cdot \Sigma^*$$

Onde $\Sigma = \{0,1\}$.
\end{erbox}

\begin{explanationbox}
\textbf{Explicação:}

A ER aceita qualquer sequência que contenha pelo menos um 0 em qualquer posição:
\begin{itemize}
    \item $(0|1)^*$: qualquer sequência de 0's e 1's antes do 0 obrigatório
    \item $0$: pelo menos um 0 (obrigatório)
    \item $(0|1)^*$: qualquer sequência de 0's e 1's depois
\end{itemize}

\textbf{Linguagem:} $L = \{w \in \{0,1\}^* \mid w \text{ contém pelo menos um } 0\}$
\end{explanationbox}

\begin{examplebox}
\textbf{Palavras que pertencem:}
\begin{itemize}
    \item $0$ (apenas um 0)
    \item $10$ (1 seguido de 0)
    \item $01$ (0 seguido de 1)
    \item $100$ (1, depois 0, depois 0)
    \item $010$ (0 no meio)
    \item $11011$ (contém 0's)
\end{itemize}

\textbf{Palavras que NÃO pertencem:}
\begin{itemize}
    \item $\varepsilon$ (vazia - não tem 0)
    \item $1$ (apenas 1's)
    \item $11$ (apenas 1's)
    \item $1111$ (apenas 1's)
\end{itemize}
\end{examplebox}

\subsection{(b) Palavras de tamanho ímpar}

\textbf{Descrição:} O conjunto das palavras de tamanho ímpar.

\begin{erbox}
\textbf{Expressão Regular:}
$$(0|1) \cdot ((0|1)(0|1))^*$$

Ou de forma mais compacta:
$$(0|1) \cdot (\Sigma\Sigma)^*$$
\end{erbox}

\begin{explanationbox}
\textbf{Explicação:}

A ER garante tamanho ímpar:
\begin{itemize}
    \item $(0|1)$: primeiro símbolo (contribui com 1 ao tamanho)
    \item $((0|1)(0|1))^*$: pares de símbolos repetidos 0 ou mais vezes
    \begin{itemize}
        \item Cada iteração adiciona 2 ao tamanho
        \item 0 iterações: tamanho = 1 (ímpar)
        \item 1 iteração: tamanho = 1 + 2 = 3 (ímpar)
        \item 2 iterações: tamanho = 1 + 4 = 5 (ímpar)
        \item $k$ iterações: tamanho = 1 + 2k (sempre ímpar)
    \end{itemize}
\end{itemize}

\textbf{Linguagem:} $L = \{w \in \{0,1\}^* \mid |w| \equiv 1 \pmod{2}\}$
\end{explanationbox}

\begin{examplebox}
\textbf{Palavras que pertencem (tamanho ímpar):}
\begin{itemize}
    \item $0$ (tamanho 1)
    \item $1$ (tamanho 1)
    \item $010$ (tamanho 3)
    \item $111$ (tamanho 3)
    \item $10101$ (tamanho 5)
    \item $0000000$ (tamanho 7)
\end{itemize}

\textbf{Palavras que NÃO pertencem (tamanho par ou vazia):}
\begin{itemize}
    \item $\varepsilon$ (tamanho 0 - par)
    \item $01$ (tamanho 2 - par)
    \item $1010$ (tamanho 4 - par)
    \item $111111$ (tamanho 6 - par)
\end{itemize}
\end{examplebox}

\subsection{(c) Palavras com tamanho múltiplo de 5}

\textbf{Descrição:} O conjunto das palavras com tamanho múltiplo de 5.

\begin{erbox}
\textbf{Expressão Regular:}
$$((0|1)(0|1)(0|1)(0|1)(0|1))^*$$

Ou de forma mais compacta:
$$(\Sigma^5)^* \text{ onde } \Sigma = \{0,1\}$$

Expandindo:
$$(00000|00001|00010|\cdots|11111)^*$$
\end{erbox}

\begin{explanationbox}
\textbf{Explicação:}

A ER aceita apenas palavras cujo tamanho é múltiplo de 5:
\begin{itemize}
    \item $(0|1)(0|1)(0|1)(0|1)(0|1)$: exatamente 5 símbolos quaisquer
    \item $(...)^*$: repete o bloco de 5 símbolos 0 ou mais vezes
    \begin{itemize}
        \item 0 repetições: $\varepsilon$ (tamanho 0 = $5 \times 0$)
        \item 1 repetição: tamanho 5 = $5 \times 1$
        \item 2 repetições: tamanho 10 = $5 \times 2$
        \item $k$ repetições: tamanho $5k$
    \end{itemize}
\end{itemize}

\textbf{Linguagem:} $L = \{w \in \{0,1\}^* \mid |w| \equiv 0 \pmod{5}\}$
\end{explanationbox}

\begin{examplebox}
\textbf{Palavras que pertencem (tamanho múltiplo de 5):}
\begin{itemize}
    \item $\varepsilon$ (tamanho 0 = $5 \times 0$)
    \item $00000$ (tamanho 5)
    \item $10101$ (tamanho 5)
    \item $0000000000$ (tamanho 10 = $5 \times 2$)
    \item $111110000011111$ (tamanho 15 = $5 \times 3$)
\end{itemize}

\textbf{Palavras que NÃO pertencem:}
\begin{itemize}
    \item $0$ (tamanho 1)
    \item $01$ (tamanho 2)
    \item $010$ (tamanho 3)
    \item $0101$ (tamanho 4)
    \item $010101$ (tamanho 6)
\end{itemize}
\end{examplebox}

\subsection{(d) Prefixo de 0's seguido de sufixo de 1's}

\textbf{Descrição:} O conjunto das palavras com um prefixo de um ou mais 0's seguido de um sufixo de zero ou mais 1's.

\begin{erbox}
\textbf{Expressão Regular:}
$$0^+ \cdot 1^*$$

Ou expandindo o $+$:
$$0 \cdot 0^* \cdot 1^*$$
\end{erbox}

\begin{explanationbox}
\textbf{Explicação:}

A ER define o padrão específico:
\begin{itemize}
    \item $0^+$: um ou mais 0's (prefixo obrigatório)
    \begin{itemize}
        \item Pelo menos um 0 deve aparecer
        \item Pode ter vários 0's consecutivos
    \end{itemize}
    \item $1^*$: zero ou mais 1's (sufixo opcional)
    \begin{itemize}
        \item Pode não ter nenhum 1
        \item Pode ter vários 1's consecutivos
    \end{itemize}
\end{itemize}

\textbf{Padrão aceito:} $\underbrace{00\cdots0}_{\geq 1} \underbrace{11\cdots1}_{\geq 0}$

\textbf{Linguagem:} $L = \{0^m 1^n \mid m \geq 1, n \geq 0\}$
\end{explanationbox}

\begin{examplebox}
\textbf{Palavras que pertencem:}
\begin{itemize}
    \item $0$ (um 0, zero 1's)
    \item $01$ (um 0, um 1)
    \item $00$ (dois 0's, zero 1's)
    \item $001$ (dois 0's, um 1)
    \item $0011$ (dois 0's, dois 1's)
    \item $000111$ (três 0's, três 1's)
\end{itemize}

\textbf{Palavras que NÃO pertencem:}
\begin{itemize}
    \item $\varepsilon$ (não tem nenhum 0)
    \item $1$ (começa com 1, não com 0)
    \item $10$ (começa com 1)
    \item $010$ (0 e 1 intercalados)
    \item $0101$ (não segue padrão $0^+1^*$)
\end{itemize}
\end{examplebox}

\subsection{(e) Palíndromos}

\textbf{Descrição:} O conjunto de palavras em que a palavra lida é igual quando lida da esquerda para direita ou vice-versa (palíndromos).

\begin{erbox}
\textbf{Resposta:}

\textbf{NÃO É POSSÍVEL} descrever linguagem de palíndromos com Expressão Regular!

Palíndromos sobre $\{0,1\}$ NÃO formam linguagem regular.
\end{erbox}

\begin{explanationbox}
\textbf{Explicação:}

A linguagem de palíndromos $L_{pal} = \{w \mid w = w^R\}$ \textbf{não é regular}.

\textbf{Prova (Lema do Bombeamento):}

Suponha que $L_{pal}$ é regular com constante $p$.

Considere $s = 0^p 1 0^p \in L_{pal}$ (é palíndromo).

Pelo lema: $s = xyz$ onde $|xy| \leq p$, $|y| > 0$.

Logo $xy$ está contido nos primeiros $p$ símbolos (todos 0's), então $y = 0^k$ com $k > 0$.

Para $i=2$: $xy^2z = 0^p 0^k 1 0^p = 0^{p+k} 1 0^p$

Mas $0^{p+k} 1 0^p$ NÃO é palíndromo (mais 0's à esquerda que à direita).

Contradição! Logo palíndromos não são regulares.

\textbf{Classificação:}
\begin{itemize}
    \item Palíndromos são LLC (Livres de Contexto)
    \item Podem ser descritos por GLC
    \item Não podem ser descritos por ER
\end{itemize}

\textbf{GLC para palíndromos:}
\begin{align*}
    S &\to 0S0 \mid 1S1 \mid 0 \mid 1 \mid \varepsilon
\end{align*}
\end{explanationbox}

\begin{examplebox}
\textbf{Exemplos de palíndromos (mas não há ER):}
\begin{itemize}
    \item $\varepsilon$ (vazia)
    \item $0$, $1$ (tamanho 1)
    \item $00$, $11$ (tamanho 2)
    \item $010$, $101$ (tamanho 3)
    \item $0110$, $1001$ (tamanho 4)
    \item $01010$, $10101$ (tamanho 5)
\end{itemize}

\textbf{Não-palíndromos:}
\begin{itemize}
    \item $01$ (reverso é 10)
    \item $10$ (reverso é 01)
    \item $001$ (reverso é 100)
    \item $0101$ (reverso é 1010)
\end{itemize}
\end{examplebox}

\subsection{(f) Primeira metade idêntica à segunda}

\textbf{Descrição:} O conjunto das palavras de tamanho par cuja primeira metade é idêntica à segunda.

\begin{erbox}
\textbf{Resposta:}

\textbf{NÃO É POSSÍVEL} descrever esta linguagem com Expressão Regular!

A linguagem $L = \{ww \mid w \in \{0,1\}^*\}$ NÃO é regular.
\end{erbox}

\begin{explanationbox}
\textbf{Explicação:}

A linguagem $L = \{ww \mid w \in \{0,1\}^*\}$ \textbf{não é regular}.

\textbf{Prova (Lema do Bombeamento):}

Suponha que $L$ é regular com constante $p$.

Considere $s = 0^p 1^p 0^p 1^p \in L$ (é $ww$ onde $w = 0^p 1^p$).

Temos $|s| = 4p \geq p$. Pelo lema: $s = xyz$ onde $|xy| \leq p$, $|y| > 0$.

Logo $xy$ está nos primeiros $p$ símbolos (todos 0's da primeira metade), então $y = 0^k$ com $k > 0$.

Para $i=2$: $xy^2z = 0^{p+k} 1^p 0^p 1^p$

Mas isso não é da forma $ww$ para nenhum $w$ (primeira metade tem $p+k$ zeros, segunda tem $p$ zeros).

Contradição! Logo $L$ não é regular.

\textbf{Classificação:}
\begin{itemize}
    \item Esta linguagem é LLC (Livre de Contexto)
    \item Pode ser descrita por GLC
    \item Não pode ser descrita por ER
\end{itemize}

\textbf{GLC para $ww$:}
\begin{align*}
    S &\to 0S0 \mid 1S1 \mid \varepsilon
\end{align*}
\end{explanationbox}

\begin{examplebox}
\textbf{Exemplos de $ww$ (mas não há ER):}
\begin{itemize}
    \item $\varepsilon$ (vazia = $\varepsilon\varepsilon$)
    \item $00$ ($w=0$, então $ww=00$)
    \item $11$ ($w=1$, então $ww=11$)
    \item $0101$ ($w=01$, então $ww=0101$)
    \item $1010$ ($w=10$, então $ww=1010$)
    \item $001001$ ($w=001$, então $ww=001001$)
\end{itemize}

\textbf{Não é $ww$:}
\begin{itemize}
    \item $0$ (tamanho ímpar)
    \item $01$ (não é repetição)
    \item $010$ (tamanho ímpar)
    \item $0011$ (não é $ww$)
    \item $0110$ (não é $ww$)
\end{itemize}
\end{examplebox}

\newpage

% ============================================================
% EXERCÍCIO 2
% ============================================================
\section{Exercício 2: ERs sobre $\Sigma=\{a,b,c\}$}

\textbf{Construa ERs para as seguintes linguagens sobre $\Sigma=\{a,b,c\}$:}

\subsection{(a) Sufixo abe ou cba}

\textbf{Linguagem:} $L_a = \{w \mid w$ contenha o sufixo $abe$ ou $cba\}$

\begin{erbox}
\textbf{Expressão Regular:}
$$(a|b|c)^* \cdot (abe | cba)$$

Ou expandindo:
$$\Sigma^* \cdot (abe | cba)$$
\end{erbox}

\begin{explanationbox}
\textbf{Explicação:}

A ER garante que a palavra termina com $abe$ ou $cba$:
\begin{itemize}
    \item $(a|b|c)^*$: qualquer sequência de símbolos antes do sufixo
    \item $(abe | cba)$: palavra deve terminar com uma dessas duas sequências
\end{itemize}
\end{explanationbox}

\begin{examplebox}
\textbf{Palavras que pertencem:}
\begin{itemize}
    \item $abe$ (apenas sufixo)
    \item $cba$ (apenas sufixo)
    \item $aabe$ (a + abe)
    \item $bcba$ (b + cba)
    \item $abcabe$ (abc + abe)
    \item $aacba$ (aa + cba)
\end{itemize}

\textbf{Palavras que NÃO pertencem:}
\begin{itemize}
    \item $\varepsilon$ (vazia)
    \item $ab$ (incompleto)
    \item $abc$ (termina com c, não com sufixo correto)
    \item $abea$ (sufixo não está no final)
    \item $cbab$ (termina com b, não com sufixo correto)
\end{itemize}
\end{examplebox}

\subsection{(b) Pelo menos 3 ocorrências de abc}

\textbf{Linguagem:} $L_b = \{w \mid w$ contenha pelo menos 3 ocorrências de $abc\}$

\begin{erbox}
\textbf{Expressão Regular:}
$$\Sigma^* \cdot abc \cdot \Sigma^* \cdot abc \cdot \Sigma^* \cdot abc \cdot \Sigma^*$$

Expandindo $\Sigma$:
$$(a|b|c)^* \cdot abc \cdot (a|b|c)^* \cdot abc \cdot (a|b|c)^* \cdot abc \cdot (a|b|c)^*$$
\end{erbox}

\begin{explanationbox}
\textbf{Explicação:}

A ER força a presença de pelo menos 3 ocorrências de $abc$:
\begin{itemize}
    \item $\Sigma^*$: símbolos arbitrários antes do primeiro $abc$
    \item $abc$: primeira ocorrência obrigatória
    \item $\Sigma^*$: símbolos arbitrários entre primeira e segunda ocorrências
    \item $abc$: segunda ocorrência obrigatória
    \item $\Sigma^*$: símbolos arbitrários entre segunda e terceira ocorrências
    \item $abc$: terceira ocorrência obrigatória
    \item $\Sigma^*$: símbolos arbitrários após a terceira ocorrência
\end{itemize}

Pode ter mais de 3 ocorrências - a ER garante \textbf{pelo menos} 3.
\end{explanationbox}

\begin{examplebox}
\textbf{Palavras que pertencem:}
\begin{itemize}
    \item $abcabcabc$ (exatamente 3 ocorrências)
    \item $aabcbabccabc$ (3 ocorrências com símbolos entre elas)
    \item $abcaabcbabc$ (3 ocorrências)
    \item $abcabcabcabc$ (4 ocorrências)
    \item $xabcyabczabc$ (onde x,y,z são sequências quaisquer)
\end{itemize}

\textbf{Palavras que NÃO pertencem:}
\begin{itemize}
    \item $\varepsilon$ (nenhuma ocorrência)
    \item $abc$ (apenas 1 ocorrência)
    \item $abcabc$ (apenas 2 ocorrências)
    \item $aabbcc$ (não contém $abc$)
    \item $abab$ (não contém $abc$)
\end{itemize}
\end{examplebox}

\subsection{(c) Último símbolo igual ao primeiro}

\textbf{Linguagem:} $L_c = \{w \mid$ o último símbolo de $w$ seja igual ao primeiro$\}$

\begin{erbox}
\textbf{Expressão Regular:}
$$a \cdot \Sigma^* \cdot a \mid b \cdot \Sigma^* \cdot b \mid c \cdot \Sigma^* \cdot c \mid a \mid b \mid c$$

Ou de forma mais compacta:
$$a(\Sigma^*a | \varepsilon) \mid b(\Sigma^*b | \varepsilon) \mid c(\Sigma^*c | \varepsilon)$$
\end{erbox}

\begin{explanationbox}
\textbf{Explicação:}

A ER cobre todos os casos onde primeiro = último:
\begin{itemize}
    \item $a \cdot \Sigma^* \cdot a$: começa com $a$, termina com $a$ (tamanho $\geq 2$)
    \item $b \cdot \Sigma^* \cdot b$: começa com $b$, termina com $b$ (tamanho $\geq 2$)
    \item $c \cdot \Sigma^* \cdot c$: começa com $c$, termina com $c$ (tamanho $\geq 2$)
    \item $a \mid b \mid c$: palavras de tamanho 1 (primeiro = último trivialmente)
\end{itemize}

O $\Sigma^*$ no meio permite qualquer sequência entre primeiro e último símbolo.
\end{explanationbox}

\begin{examplebox}
\textbf{Palavras que pertencem:}
\begin{itemize}
    \item $a$ (tamanho 1)
    \item $b$, $c$ (tamanho 1)
    \item $aa$ (começa e termina com $a$)
    \item $aba$ (começa e termina com $a$)
    \item $bab$ (começa e termina com $b$)
    \item $cabac$ (começa e termina com $c$)
    \item $abcba$ (começa e termina com $a$)
\end{itemize}

\textbf{Palavras que NÃO pertencem:}
\begin{itemize}
    \item $\varepsilon$ (sem símbolos)
    \item $ab$ (começa com $a$, termina com $b$)
    \item $ba$ (começa com $b$, termina com $a$)
    \item $abc$ (começa com $a$, termina com $c$)
    \item $cab$ (começa com $c$, termina com $b$)
\end{itemize}
\end{examplebox}

\subsection{(d) 2 a's ou 2 b's consecutivos}

\textbf{Linguagem:} $L_d = \{w \mid w$ tenha 2 $a$'s consecutivos ou 2 $b$'s consecutivos$\}$

\begin{erbox}
\textbf{Expressão Regular:}
$$\Sigma^* \cdot (aa | bb) \cdot \Sigma^*$$

Expandindo:
$$(a|b|c)^* \cdot (aa | bb) \cdot (a|b|c)^*$$
\end{erbox}

\begin{explanationbox}
\textbf{Explicação:}

A ER garante a presença de pelo menos uma das duas condições:
\begin{itemize}
    \item $\Sigma^*$: qualquer sequência antes dos consecutivos
    \item $(aa | bb)$: deve conter $aa$ OU $bb$ em algum lugar
    \item $\Sigma^*$: qualquer sequência depois dos consecutivos
\end{itemize}

Basta ter $aa$ ou $bb$ em qualquer posição da palavra.
\end{explanationbox}

\begin{examplebox}
\textbf{Palavras que pertencem:}
\begin{itemize}
    \item $aa$ (dois $a$'s consecutivos)
    \item $bb$ (dois $b$'s consecutivos)
    \item $aab$ (começa com $aa$)
    \item $bba$ (começa com $bb$)
    \item $caa$ (termina com $aa$)
    \item $cbb$ (termina com $bb$)
    \item $caac$ ($aa$ no meio)
    \item $abbc$ ($bb$ no meio)
    \item $aabb$ (tem ambos $aa$ e $bb$)
\end{itemize}

\textbf{Palavras que NÃO pertencem:}
\begin{itemize}
    \item $\varepsilon$ (vazia)
    \item $a$, $b$, $c$ (apenas um símbolo)
    \item $ab$ (não tem consecutivos)
    \item $abc$ (nenhum símbolo se repete)
    \item $abab$ (a's e b's alternados)
    \item $ababc$ (sem consecutivos)
    \item $cabac$ (sem $aa$ nem $bb$)
\end{itemize}
\end{examplebox}

\subsection{(e) aa ou bb é subpalavra e cccc é sufixo}

\textbf{Linguagem:} $L_e = \{w \mid aa$ ou $bb$ é subpalavra e $cccc$ é sufixo$\}$

\begin{erbox}
\textbf{Expressão Regular:}
$$\Sigma^* \cdot (aa | bb) \cdot \Sigma^* \cdot cccc$$

Expandindo:
$$(a|b|c)^* \cdot (aa | bb) \cdot (a|b|c)^* \cdot cccc$$
\end{erbox}

\begin{explanationbox}
\textbf{Explicação:}

A ER combina duas condições obrigatórias:
\begin{enumerate}
    \item \textbf{Contém $aa$ ou $bb$ como subpalavra:}
    \begin{itemize}
        \item $\Sigma^* \cdot (aa | bb) \cdot \Sigma^*$
        \item $aa$ ou $bb$ pode aparecer em qualquer posição
    \end{itemize}
    
    \item \textbf{Termina com $cccc$:}
    \begin{itemize}
        \item $cccc$ no final é obrigatório
        \item São exatamente 4 $c$'s consecutivos ao final
    \end{itemize}
\end{enumerate}

Ambas as condições devem ser satisfeitas simultaneamente.
\end{explanationbox}

\begin{examplebox}
\textbf{Palavras que pertencem:}
\begin{itemize}
    \item $aacccc$ (tem $aa$ e termina com $cccc$)
    \item $bbcccc$ (tem $bb$ e termina com $cccc$)
    \item $caacccc$ (tem $aa$ e termina com $cccc$)
    \item $cbbcccc$ (tem $bb$ e termina com $cccc$)
    \item $aabbcccc$ (tem $aa$ e $bb$, termina com $cccc$)
    \item $abacccc$ (tem $aa$, termina com $cccc$)
\end{itemize}

\textbf{Palavras que NÃO pertencem:}
\begin{itemize}
    \item $aaccc$ (tem $aa$ mas só 3 $c$'s no final)
    \item $bbccccc$ (tem $bb$ mas 5 $c$'s no final)
    \item $abcccc$ (termina com $cccc$ mas sem $aa$ nem $bb$)
    \item $aabbc$ (tem $aa$ e $bb$ mas não termina com $cccc$)
    \item $cccc$ (termina com $cccc$ mas sem $aa$ nem $bb$)
\end{itemize}
\end{examplebox}

\subsection{(f) Contém ab e ba em qualquer ordem}

\textbf{Linguagem:} $L_f = \{w \mid w$ contenha as substrings $ab$ e $ba$ (em qualquer ordem)$\}$

\begin{erbox}
\textbf{Expressão Regular:}
$$\Sigma^* \cdot ab \cdot \Sigma^* \cdot ba \cdot \Sigma^* \mid \Sigma^* \cdot ba \cdot \Sigma^* \cdot ab \cdot \Sigma^*$$

Ou de forma mais compacta:
$$\Sigma^*(ab\Sigma^*ba | ba\Sigma^*ab)\Sigma^*$$
\end{erbox}

\begin{explanationbox}
\textbf{Explicação:}

A ER captura as duas ordens possíveis:
\begin{enumerate}
    \item \textbf{Primeiro $ab$, depois $ba$:}
    \begin{itemize}
        \item $\Sigma^* \cdot ab \cdot \Sigma^* \cdot ba \cdot \Sigma^*$
        \item $ab$ aparece antes de $ba$ (com símbolos entre eles ou não)
    \end{itemize}
    
    \item \textbf{Primeiro $ba$, depois $ab$:}
    \begin{itemize}
        \item $\Sigma^* \cdot ba \cdot \Sigma^* \cdot ab \cdot \Sigma^*$
        \item $ba$ aparece antes de $ab$ (com símbolos entre eles ou não)
    \end{itemize}
\end{enumerate}

A união ($|$) das duas alternativas garante que ambas as substrings apareçam, independentemente da ordem.
\end{explanationbox}

\begin{examplebox}
\textbf{Palavras que pertencem:}
\begin{itemize}
    \item $abba$ ($ab$ seguido de $ba$)
    \item $baab$ ($ba$ seguido de $ab$)
    \item $abcba$ ($ab$, depois $c$, depois $ba$)
    \item $bacab$ ($ba$, depois $c$, depois $ab$)
    \item $aba$ (contém $ab$ e $ba$ sobrepostos)
    \item $bab$ (contém $ba$ e $ab$ sobrepostos)
    \item $aabba$ (contém ambos)
    \item $cabbac$ (contém ambos)
\end{itemize}

\textbf{Palavras que NÃO pertencem:}
\begin{itemize}
    \item $\varepsilon$ (vazia)
    \item $ab$ (só tem $ab$, falta $ba$)
    \item $ba$ (só tem $ba$, falta $ab$)
    \item $aaa$ (não tem $ab$ nem $ba$)
    \item $abc$ (tem $ab$ mas não tem $ba$)
    \item $bac$ (tem $ba$ mas não tem $ab$)
    \item $aabbcc$ (tem $ab$ mas não tem $ba$)
\end{itemize}
\end{examplebox}

\subsection{(g) Palavras xyz com $|x|=3$ e $|z|=3$}

\textbf{Linguagem:} $L_g = \{xyz \mid x,y,z \in \Sigma^*$ e $|x|=3$ e $|z|=3\}$

\begin{erbox}
\textbf{Expressão Regular:}
$$\Sigma^3 \cdot \Sigma^* \cdot \Sigma^3$$

Expandindo:
$$(a|b|c)(a|b|c)(a|b|c) \cdot (a|b|c)^* \cdot (a|b|c)(a|b|c)(a|b|c)$$
\end{erbox}

\begin{explanationbox}
\textbf{Explicação:}

A ER garante tamanho mínimo de 6 com restrições específicas:
\begin{itemize}
    \item $\Sigma^3$: exatamente 3 símbolos no início (parte $x$)
    \item $\Sigma^*$: zero ou mais símbolos no meio (parte $y$)
    \item $\Sigma^3$: exatamente 3 símbolos no final (parte $z$)
\end{itemize}

\textbf{Tamanho mínimo:} $|w| \geq 6$ (quando $y = \varepsilon$)

\textbf{Linguagem:} Todas as palavras com tamanho $\geq 6$ sobre $\Sigma = \{a,b,c\}$
\end{explanationbox}

\begin{examplebox}
\textbf{Palavras que pertencem ($|w| \geq 6$):}
\begin{itemize}
    \item $abcabc$ (tamanho 6: $x=abc$, $y=\varepsilon$, $z=abc$)
    \item $aaabbb$ (tamanho 6: $x=aaa$, $y=\varepsilon$, $z=bbb$)
    \item $abcdefg$ (tamanho 7: $x=abc$, $y=d$, $z=efg$)
    \item $abcXYZdef$ (tamanho 9: $x=abc$, $y=XYZ$, $z=def$)
    \item $aaaaaaaaaa$ (tamanho 10: $x=aaa$, $y=aaaa$, $z=aaa$)
\end{itemize}

\textbf{Palavras que NÃO pertencem ($|w| < 6$):}
\begin{itemize}
    \item $\varepsilon$ (tamanho 0)
    \item $abc$ (tamanho 3)
    \item $abcd$ (tamanho 4)
    \item $abcde$ (tamanho 5)
\end{itemize}
\end{examplebox}

\subsection{(h) Exatamente um a}

\textbf{Linguagem:} $L_h = \{w \mid w$ contenha exatamente um $a\}$

\begin{erbox}
\textbf{Expressão Regular:}
$$(b|c)^* \cdot a \cdot (b|c)^*$$
\end{erbox}

\begin{explanationbox}
\textbf{Explicação:}

A ER garante exatamente uma ocorrência de $a$:
\begin{itemize}
    \item $(b|c)^*$: zero ou mais símbolos $b$ ou $c$ antes do $a$
    \item $a$: exatamente um $a$ (único e obrigatório)
    \item $(b|c)^*$: zero ou mais símbolos $b$ ou $c$ depois do $a$
\end{itemize}

\textbf{Restrição importante:} Não pode haver nenhum outro $a$ na palavra.
\end{explanationbox}

\begin{examplebox}
\textbf{Palavras que pertencem:}
\begin{itemize}
    \item $a$ (apenas o $a$)
    \item $ba$ ($b$ seguido de $a$)
    \item $ab$ ($a$ seguido de $b$)
    \item $bac$ ($b$, $a$, $c$)
    \item $bbabb$ (um $a$ cercado por $b$'s)
    \item $ccaccc$ (um $a$ cercado por $c$'s)
    \item $bcabc$ (um $a$ no meio)
\end{itemize}

\textbf{Palavras que NÃO pertencem:}
\begin{itemize}
    \item $\varepsilon$ (nenhum $a$)
    \item $bc$ (nenhum $a$)
    \item $aa$ (dois $a$'s)
    \item $aba$ (dois $a$'s)
    \item $aaa$ (três $a$'s)
    \item $abac$ (dois $a$'s)
\end{itemize}
\end{examplebox}

\subsection{(i) Não contém dois a's adjacentes}

\textbf{Linguagem:} $L_i = \{w \mid w$ não contém dois $a$'s adjacentes$\}$

\begin{erbox}
\textbf{Expressão Regular:}
$$(b|c)^* \cdot (a(b|c)^+)^* \cdot a^? \cdot (b|c)^*$$

Ou de forma mais simples:
$$((b|c)^* a)^* (b|c)^*$$
\end{erbox}

\begin{explanationbox}
\textbf{Explicação:}

A ER garante que entre quaisquer dois $a$'s há pelo menos um $b$ ou $c$:
\begin{itemize}
    \item $((b|c)^* a)^*$: repete o padrão "símbolos não-$a$ seguidos de um $a$"
    \begin{itemize}
        \item Cada $a$ é precedido por zero ou mais $b$'s ou $c$'s
        \item Garante que $a$'s não ficam adjacentes
    \end{itemize}
    \item $(b|c)^*$: símbolos finais (sem $a$ ou terminando antes de $a$ adjacente)
\end{itemize}

\textbf{Padrão proibido:} $aa$ não pode aparecer em nenhum lugar.
\end{explanationbox}

\begin{examplebox}
\textbf{Palavras que pertencem:}
\begin{itemize}
    \item $\varepsilon$ (vazia - sem $aa$)
    \item $a$ (só um $a$)
    \item $abc$ (um $a$ sem adjacente)
    \item $aba$ ($a$'s separados por $b$)
    \item $aca$ ($a$'s separados por $c$)
    \item $bacabaca$ ($a$'s sempre separados)
    \item $bbbccc$ (sem nenhum $a$)
    \item $abcabcabc$ ($a$'s separados)
\end{itemize}

\textbf{Palavras que NÃO pertencem:}
\begin{itemize}
    \item $aa$ (dois $a$'s adjacentes)
    \item $aab$ ($aa$ no início)
    \item $baa$ ($aa$ no final)
    \item $baab$ ($aa$ no meio)
    \item $aaa$ (múltiplos $a$'s adjacentes)
    \item $abaa$ ($aa$ no final)
\end{itemize}
\end{examplebox}

\subsection{(j) Número par de substrings ba}

\textbf{Linguagem:} $L_j = \{w \mid w$ contenha um número par de substrings $ba\}$

\begin{erbox}
\textbf{Expressão Regular:}

Esta é uma linguagem regular complexa que requer AFD com estados para rastrear paridade.

\textbf{Abordagem simplificada (incompleta):}
$$(a|b|c)^* \text{ (com restrições de paridade)}$$

\textbf{Solução via AFD:} Construir AFD com 2 estados (par/ímpar) e converter para ER.
\end{erbox}

\begin{explanationbox}
\textbf{Explicação:}

Esta linguagem é regular mas sua ER direta é muito complexa. A melhor abordagem é:

\begin{enumerate}
    \item \textbf{Construir AFD:}
    \begin{itemize}
        \item Estado $q_{par}$: número par de $ba$ vistos (aceita)
        \item Estado $q_{ímpar}$: número ímpar de $ba$ vistos
        \item Transições rastreiam quando $ba$ é formado
    \end{itemize}
    
    \item \textbf{Converter AFD para ER:}
    \begin{itemize}
        \item Usar algoritmo de eliminação de estados
        \item ER resultante será muito longa
    \end{itemize}
\end{enumerate}

\textbf{Desafio:} Contar ocorrências de substring requer rastrear contexto (último símbolo visto).

\textbf{Exemplo de lógica:}
\begin{itemize}
    \item Se último símbolo foi $b$ e lemos $a$: incrementa contador
    \item Aceitamos quando contador é par (0, 2, 4, ...)
\end{itemize}
\end{explanationbox}

\begin{examplebox}
\textbf{Palavras que pertencem (par de $ba$):}
\begin{itemize}
    \item $\varepsilon$ (0 ocorrências de $ba$ - par)
    \item $abc$ (0 ocorrências)
    \item $baba$ (2 ocorrências: $ba$ e $ba$)
    \item $ababab$ (3 $ba$'s... não, conta sobreposições)
    \item $bacba$ (2 ocorrências separadas)
\end{itemize}

\textbf{Observação:} A contagem exata depende se substrings podem se sobrepor.

\textbf{Palavras que NÃO pertencem (ímpar de $ba$):}
\begin{itemize}
    \item $ba$ (1 ocorrência)
    \item $bac$ (1 ocorrência)
    \item $aba$ (1 ocorrência)
\end{itemize}
\end{examplebox}

\subsection{(k) Não contém aa nem bb como subpalavras}

\textbf{Linguagem:} $L_k = \{w \mid w$ não contém $aa$ nem $bb$ como subpalavras$\}$

\begin{erbox}
\textbf{Expressão Regular:}
$$(c^* a c^* b)^* c^* a^? \mid (c^* b c^* a)^* c^* b^?$$

Ou de forma mais compacta:
$$c^* (ac^*b)^* (a|b|\varepsilon) c^* \mid c^* (bc^*a)^* (a|b|\varepsilon) c^*$$
\end{erbox}

\begin{explanationbox}
\textbf{Explicação:}

A ER garante alternância forçada entre $a$ e $b$:
\begin{itemize}
    \item $c$'s podem aparecer livremente em qualquer quantidade
    \item $a$'s e $b$'s devem alternar (nunca $aa$ ou $bb$)
    \item Padrões válidos:
    \begin{itemize}
        \item $a$ seguido de $b$ (com $c$'s opcionais entre/ao redor)
        \item $b$ seguido de $a$ (com $c$'s opcionais entre/ao redor)
        \item Apenas $c$'s
        \item No máximo um $a$ ou um $b$ sem repetição
    \end{itemize}
\end{itemize}

\textbf{Restrições:}
\begin{itemize}
    \item Proibido: $aa$, $bb$
    \item Permitido: $ab$, $ba$, $aba$, $bab$, etc. (com $c$'s)
\end{itemize}
\end{explanationbox}

\begin{examplebox}
\textbf{Palavras que pertencem:}
\begin{itemize}
    \item $\varepsilon$ (vazia)
    \item $c$, $cc$, $ccc$ (apenas $c$'s)
    \item $a$, $b$ (um único símbolo)
    \item $ab$, $ba$ ($a$ e $b$ alternados)
    \item $aba$, $bab$ (alternância)
    \item $cab$, $abc$, $bca$ ($c$'s misturados)
    \item $cacbcac$ ($a$ e $b$ separados por $c$'s)
    \item $abacaba$ (espere, tem $aa$... não válido)
    \item $acabcac$ (válido se não tiver $aa$ ou $bb$)
\end{itemize}

\textbf{Palavras que NÃO pertencem:}
\begin{itemize}
    \item $aa$ (proibido)
    \item $bb$ (proibido)
    \item $aab$ (contém $aa$)
    \item $bba$ (contém $bb$)
    \item $aabc$ (contém $aa$)
    \item $cbbc$ (contém $bb$)
    \item $abaa$ (contém $aa$)
\end{itemize}
\end{examplebox}

\subsection{(l) Quarto símbolo é um a}

\textbf{Linguagem:} $L_l = \{w \mid$ o quarto símbolo da esquerda para a direita é um $a\}$

\begin{erbox}
\textbf{Expressão Regular:}
$$\Sigma^3 \cdot a \cdot \Sigma^*$$

Expandindo:
$$(a|b|c)(a|b|c)(a|b|c) \cdot a \cdot (a|b|c)^*$$
\end{erbox}

\begin{explanationbox}
\textbf{Explicação:}

A ER força o quarto símbolo a ser $a$:
\begin{itemize}
    \item $\Sigma^3$: exatamente 3 símbolos quaisquer (posições 1, 2, 3)
    \item $a$: quarto símbolo deve ser $a$ (posição 4)
    \item $\Sigma^*$: zero ou mais símbolos quaisquer após a posição 4
\end{itemize}

\textbf{Tamanho mínimo:} $|w| \geq 4$ (quatro símbolos ou mais)

\textbf{Padrão:} $\_ \_ \_ a \ldots$ (qualquer, qualquer, qualquer, $a$, qualquer...)
\end{explanationbox}

\begin{examplebox}
\textbf{Palavras que pertencem:}
\begin{itemize}
    \item $bbba$ (quarto é $a$, tamanho 4)
    \item $ccca$ (quarto é $a$, tamanho 4)
    \item $abcab$ (quarto é $a$)
    \item $bbbabbb$ (quarto é $a$)
    \item $123a567$ (notação: posições 1,2,3 quaisquer, 4 é $a$)
    \item $aaaaaaaa$ (quarto é $a$)
\end{itemize}

\textbf{Palavras que NÃO pertencem:}
\begin{itemize}
    \item $\varepsilon$ (tamanho 0 $<$ 4)
    \item $a$ (tamanho 1 $<$ 4)
    \item $abc$ (tamanho 3 $<$ 4)
    \item $abcb$ (quarto é $b$, não $a$)
    \item $abcc$ (quarto é $c$, não $a$)
    \item $bbbbc$ (quarto é $b$, não $a$)
\end{itemize}
\end{examplebox}

\newpage

% ============================================================
% EXERCÍCIO 3
% ============================================================
\section{Exercício 3: Análise de ERs}

\textbf{Dado o alfabeto $\Sigma=\{a,b\}$. Para cada uma das linguagens a seguir, representadas na forma de expressões regulares, apresente pelo menos duas palavras que pertençam à linguagem e duas que não pertençam:}

\subsection{(a) $a^* b^*$}

\begin{erbox}
\textbf{Expressão Regular:} $a^* b^*$

\textbf{Linguagem:} $L = \{a^m b^n \mid m,n \geq 0\}$

Palavras formadas por zero ou mais $a$'s seguidos de zero ou mais $b$'s.
\end{erbox}

\begin{explanationbox}
\textbf{Explicação:}

A ER aceita palavras no padrão: $\underbrace{aa\cdots a}_{m \geq 0} \underbrace{bb\cdots b}_{n \geq 0}$

\textbf{Características:}
\begin{itemize}
    \item Todos os $a$'s (se houver) aparecem antes de todos os $b$'s (se houver)
    \item Pode ter apenas $a$'s, apenas $b$'s, ambos, ou nenhum (vazia)
    \item NÃO permite intercalação de $a$'s e $b$'s
\end{itemize}
\end{explanationbox}

\begin{acceptbox}
\textbf{Duas (ou mais) palavras que PERTENCEM à linguagem:}
\begin{enumerate}
    \item $\varepsilon$ (cadeia vazia: $a^0 b^0$)
    \item $a$ (apenas um $a$: $a^1 b^0$)
    \item $b$ (apenas um $b$: $a^0 b^1$)
    \item $ab$ ($a^1 b^1$)
    \item $aab$ ($a^2 b^1$)
    \item $abb$ ($a^1 b^2$)
    \item $aaabbb$ ($a^3 b^3$)
    \item $aaaaa$ (apenas $a$'s: $a^5 b^0$)
    \item $bbbbb$ (apenas $b$'s: $a^0 b^5$)
\end{enumerate}
\end{acceptbox}

\begin{rejectbox}
\textbf{Duas (ou mais) palavras que NÃO PERTENCEM à linguagem:}
\begin{enumerate}
    \item $ba$ ($b$ seguido de $a$ - ordem errada)
    \item $aba$ ($a$'s e $b$'s intercalados)
    \item $abab$ (intercalação)
    \item $bba$ (tem $b$ antes de $a$)
    \item $aabba$ (tem $bb$ seguido de $a$)
    \item $baaab$ (começa com $b$, depois $a$'s)
\end{enumerate}

\textbf{Razão:} Todas violam o padrão $a^* b^*$ por terem $b$ antes de $a$ ou intercalação.
\end{rejectbox}

\subsection{(b) $a(ba)^* b$}

\begin{erbox}
\textbf{Expressão Regular:} $a(ba)^* b$

\textbf{Linguagem:} $L = \{a(ba)^n b \mid n \geq 0\}$

Palavras que começam com $a$, terminam com $b$, e têm zero ou mais repetições de $ba$ no meio.
\end{erbox}

\begin{explanationbox}
\textbf{Explicação:}

A ER define o padrão: $a \underbrace{baba\cdots ba}_{n \text{ vezes}} b$

\textbf{Características:}
\begin{itemize}
    \item Sempre começa com $a$ (obrigatório)
    \item Sempre termina com $b$ (obrigatório)
    \item No meio: zero ou mais repetições do bloco $ba$
    \item Tamanho mínimo: 2 (quando $n=0$: $ab$)
    \item Tamanho geral: $2 + 2n$ (sempre par)
\end{itemize}

\textbf{Padrão de tamanhos:}
\begin{itemize}
    \item $n=0$: $ab$ (tamanho 2)
    \item $n=1$: $abab$ (tamanho 4)
    \item $n=2$: $ababab$ (tamanho 6)
    \item $n=k$: tamanho $2+2k$
\end{itemize}
\end{explanationbox}

\begin{acceptbox}
\textbf{Duas (ou mais) palavras que PERTENCEM à linguagem:}
\begin{enumerate}
    \item $ab$ (caso base: $n=0$)
    \item $abab$ (uma repetição: $n=1$)
    \item $ababab$ (duas repetições: $n=2$)
    \item $abababab$ (três repetições: $n=3$)
    \item $ababababab$ (quatro repetições: $n=4$)
\end{enumerate}
\end{acceptbox}

\begin{rejectbox}
\textbf{Duas (ou mais) palavras que NÃO PERTENCEM à linguagem:}
\begin{enumerate}
    \item $\varepsilon$ (vazia - não começa com $a$ nem termina com $b$)
    \item $a$ (não termina com $b$)
    \item $b$ (não começa com $a$)
    \item $ba$ (ordem errada - começa com $b$)
    \item $aba$ (termina com $a$, não com $b$)
    \item $bab$ (começa com $b$, não com $a$)
    \item $aab$ (tem $aa$ - não segue padrão $(ba)^*$)
    \item $abb$ (tem $bb$ - não segue padrão $(ba)^*$)
    \item $aabb$ (não segue padrão $a(ba)^*b$)
\end{enumerate}

\textbf{Razão:} Violam início com $a$, fim com $b$, ou padrão $(ba)^*$ no meio.
\end{rejectbox}

\subsection{(c) $(aaa)^*$}

\begin{erbox}
\textbf{Expressão Regular:} $(aaa)^*$

\textbf{Linguagem:} $L = \{(aaa)^n \mid n \geq 0\} = \{a^{3n} \mid n \geq 0\}$

Palavras formadas por zero ou mais repetições de $aaa$ (tamanho múltiplo de 3).
\end{erbox}

\begin{explanationbox}
\textbf{Explicação:}

A ER aceita apenas palavras compostas exclusivamente por $a$'s cujo tamanho é múltiplo de 3:

\textbf{Características:}
\begin{itemize}
    \item Apenas símbolo $a$ (nenhum $b$)
    \item Tamanho deve ser $3n$ para algum $n \geq 0$
    \item Tamanhos válidos: 0, 3, 6, 9, 12, 15, ...
\end{itemize}

\textbf{Exemplos por tamanho:}
\begin{itemize}
    \item $n=0$: $\varepsilon$ (tamanho 0)
    \item $n=1$: $aaa$ (tamanho 3)
    \item $n=2$: $aaaaaa$ (tamanho 6)
    \item $n=3$: $aaaaaaaaa$ (tamanho 9)
\end{itemize}
\end{explanationbox}

\begin{acceptbox}
\textbf{Duas (ou mais) palavras que PERTENCEM à linguagem:}
\begin{enumerate}
    \item $\varepsilon$ (vazia: $(aaa)^0$, tamanho 0 = $3 \times 0$)
    \item $aaa$ (uma repetição: $(aaa)^1$, tamanho 3)
    \item $aaaaaa$ (duas repetições: $(aaa)^2$, tamanho 6)
    \item $aaaaaaaaa$ (três repetições: $(aaa)^3$, tamanho 9)
    \item $aaaaaaaaaaaa$ (quatro repetições: $(aaa)^4$, tamanho 12)
\end{enumerate}
\end{acceptbox}

\begin{rejectbox}
\textbf{Duas (ou mais) palavras que NÃO PERTENCEM à linguagem:}
\begin{enumerate}
    \item $a$ (tamanho 1 - não é múltiplo de 3)
    \item $aa$ (tamanho 2 - não é múltiplo de 3)
    \item $aaaa$ (tamanho 4 - não é múltiplo de 3)
    \item $aaaaa$ (tamanho 5 - não é múltiplo de 3)
    \item $b$ (contém $b$ - só permite $a$)
    \item $ab$ (contém $b$)
    \item $aaab$ (contém $b$)
    \item $aaaaaaa$ (tamanho 7 - não é múltiplo de 3)
\end{enumerate}

\textbf{Razão:} Tamanho não é múltiplo de 3 ou contém símbolo diferente de $a$.
\end{rejectbox}

\subsection{(d) $aba \cup baba$}

\begin{erbox}
\textbf{Expressão Regular:} $aba \cup baba$ (ou $aba | baba$)

\textbf{Linguagem:} $L = \{aba, baba\}$

Linguagem finita contendo exatamente duas palavras.
\end{erbox}

\begin{explanationbox}
\textbf{Explicação:}

A ER define uma linguagem \textbf{finita} com apenas 2 elementos:

\textbf{Características:}
\begin{itemize}
    \item União ($\cup$ ou $|$) de duas cadeias específicas
    \item Apenas duas palavras são aceitas: $aba$ e $baba$
    \item Qualquer outra palavra é rejeitada
    \item Tamanhos: $aba$ tem tamanho 3, $baba$ tem tamanho 4
\end{itemize}

\textbf{Importante:} A união não gera novas palavras, apenas aceita as duas especificadas.
\end{explanationbox}

\begin{acceptbox}
\textbf{Duas palavras que PERTENCEM à linguagem:}
\begin{enumerate}
    \item $aba$ (primeira alternativa da união)
    \item $baba$ (segunda alternativa da união)
\end{enumerate}

\textbf{Observação:} Estas são as \textbf{únicas} duas palavras da linguagem!
\end{acceptbox}

\begin{rejectbox}
\textbf{Duas (ou mais) palavras que NÃO PERTENCEM à linguagem:}
\begin{enumerate}
    \item $\varepsilon$ (vazia)
    \item $a$ (diferente de $aba$ e $baba$)
    \item $b$ (diferente de $aba$ e $baba$)
    \item $ab$ (prefixo de $aba$ mas não é $aba$)
    \item $ba$ (prefixo de $baba$ mas não é $baba$)
    \item $abab$ (parece mistura mas não é exatamente $aba$ nem $baba$)
    \item $aaba$ (ordem errada)
    \item $abba$ (diferente)
    \item $bababa$ (mais longo que $baba$)
    \item Literalmente qualquer outra palavra diferente de $aba$ ou $baba$
\end{enumerate}

\textbf{Razão:} A linguagem contém APENAS as palavras $aba$ e $baba$, nada mais.
\end{rejectbox}

\newpage

% ============================================================
% OBSERVAÇÕES FINAIS
% ============================================================
\section{Observações Finais}

\subsection{Resumo Completo dos Exercícios}

\begin{center}
\begin{tabular}{|c|l|c|}
\hline
\textbf{Exercício} & \textbf{Linguagem} & \textbf{Regular?} \\
\hline
1a & Contém pelo menos um 0 & Sim \\
1b & Tamanho ímpar & Sim \\
1c & Tamanho múltiplo de 5 & Sim \\
1d & Padrão $0^+1^*$ & Sim \\
1e & Palíndromos & \textbf{NÃO} (LLC) \\
1f & $ww$ (repetição) & \textbf{NÃO} (LLC) \\
\hline
2a & Sufixo $abe$ ou $cba$ & Sim \\
2b & $\geq$ 3 ocorrências de $abc$ & Sim \\
2c & Primeiro = último & Sim \\
2d & Contém $aa$ ou $bb$ & Sim \\
2e & ($aa$ ou $bb$) e sufixo $cccc$ & Sim \\
2f & Contém $ab$ e $ba$ & Sim \\
2g & $|x|=3, |z|=3$ ($|w| \geq 6$) & Sim \\
2h & Exatamente um $a$ & Sim \\
2i & Sem $aa$ adjacente & Sim \\
2j & Par de substrings $ba$ & Sim (complexa) \\
2k & Sem $aa$ nem $bb$ & Sim \\
2l & Quarto símbolo é $a$ & Sim \\
\hline
3a & $a^*b^*$ & Sim \\
3b & $a(ba)^*b$ & Sim \\
3c & $(aaa)^*$ & Sim \\
3d & $aba \cup baba$ & Sim (finita) \\
\hline
\end{tabular}
\end{center}

\subsection{Conceitos-Chave Abordados}

\begin{itemize}
    \item \textbf{Operadores básicos de ER:}
    \begin{itemize}
        \item Concatenação: $AB$ (A seguido de B)
        \item União: $A|B$ ou $A \cup B$ (A ou B)
        \item Fecho de Kleene: $A^*$ (zero ou mais A's)
        \item Fecho positivo: $A^+$ (um ou mais A's) = $AA^*$
    \end{itemize}
    
    \item \textbf{Padrões comuns:}
    \begin{itemize}
        \item Prefixo: $padr\tilde{a}o \cdot \Sigma^*$
        \item Sufixo: $\Sigma^* \cdot padr\tilde{a}o$
        \item Subpalavra: $\Sigma^* \cdot padr\tilde{a}o \cdot \Sigma^*$
        \item Tamanho fixo $n$: $\Sigma^n$
        \item Tamanho ímpar: $\Sigma(\Sigma\Sigma)^*$
        \item Tamanho múltiplo de $k$: $(\Sigma^k)^*$
        \item Primeira = última: $a\Sigma^*a | b\Sigma^*b | \cdots | a | b | \cdots$
    \end{itemize}
    
    \item \textbf{Negações e restrições:}
    \begin{itemize}
        \item Sem símbolo $s$: usar apenas outros símbolos
        \item Sem substring $xy$: garantir alternância ou separação
        \item Exatamente $k$ ocorrências: padrão específico
    \end{itemize}
    
    \item \textbf{Limitações de ERs:}
    \begin{itemize}
        \item Não expressam palíndromos (requer LLC)
        \item Não expressam $ww$ (requer LLC)
        \item Não expressam $a^nb^n$ (requer LLC)
        \item Não conseguem contar arbitrariamente
        \item Limitadas a linguagens regulares
    \end{itemize}
    
    \item \textbf{Lema do Bombeamento:}
    \begin{itemize}
        \item Ferramenta para provar que linguagem NÃO é regular
        \item Aplicado em palíndromos e $ww$
        \item Contradição via bombeamento de substring
    \end{itemize}
    
    \item \textbf{Tipos de linguagens regulares:}
    \begin{itemize}
        \item Finitas: conjunto fixo de palavras (ex: $\{aba, baba\}$)
        \item Infinitas: padrões repetitivos (ex: $a^*b^*$, $(aaa)^*$)
        \item Com restrições: tamanho, posição, ocorrências
    \end{itemize}
\end{itemize}

\subsection{Técnicas de Construção de ERs}

\begin{enumerate}
    \item \textbf{Análise da linguagem:}
    \begin{itemize}
        \item Identificar padrões obrigatórios
        \item Identificar partes opcionais
        \item Identificar ordem e estrutura
    \end{itemize}
    
    \item \textbf{Decomposição:}
    \begin{itemize}
        \item Dividir em prefixo, meio, sufixo
        \item Usar união para alternativas
        \item Usar concatenação para sequências
    \end{itemize}
    
    \item \textbf{Casos especiais:}
    \begin{itemize}
        \item Cadeia vazia: incluir $\varepsilon$ ou usar $^*$
        \item Tamanho 1: símbolos individuais
        \item Tamanho fixo: contar símbolos
    \end{itemize}
    
    \item \textbf{Verificação:}
    \begin{itemize}
        \item Testar com exemplos pequenos
        \item Verificar casos extremos (vazio, tamanho 1)
        \item Confirmar contra-exemplos
    \end{itemize}
\end{enumerate}

\subsection{Aplicações Práticas}

\begin{itemize}
    \item \textbf{Análise léxica:} Tokenização em compiladores
    \item \textbf{Validação de entrada:} Formatos de email, telefone, datas
    \item \textbf{Busca de texto:} grep, sed, editores de texto
    \item \textbf{Processamento de strings:} Parsing simples
    \item \textbf{Redes:} Filtros de pacotes, padrões de tráfego
\end{itemize}

\end{document}
