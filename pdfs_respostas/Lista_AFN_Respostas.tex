% !TEX program = pdflatex
\documentclass[a4paper,12pt]{article}

% ================== PACOTES ==================
\usepackage[utf8]{inputenc}
\usepackage[T1]{fontenc}
\usepackage[brazilian]{babel}
\usepackage{amsmath,amssymb,amsthm}
\usepackage{graphicx}
\usepackage{float}
\usepackage{hyperref}
\usepackage{fancyhdr}
\usepackage{geometry}
\usepackage{tcolorbox}
\usepackage{enumitem}
\usepackage{booktabs}
\usepackage{multirow}
\usepackage{array}
\usepackage{xcolor}
\usepackage{tikz}

% ================== CONFIGURAÇÕES ==================
\geometry{
    left=2.5cm,
    right=2.5cm,
    top=2.5cm,
    bottom=2.5cm
}

\hypersetup{
    colorlinks=true,
    linkcolor=blue,
    filecolor=magenta,
    urlcolor=cyan,
}

% ================== CABEÇALHO/RODAPÉ ==================
\pagestyle{fancy}
\fancyhf{}
\fancyhead[L]{LFA - Linguagens Formais e Autômatos}
\fancyhead[R]{CEFET-MG}
\fancyfoot[C]{\thepage}
\renewcommand{\headrulewidth}{0.4pt}
\renewcommand{\footrulewidth}{0.4pt}

% ================== BOXES ==================
\newtcolorbox{enunciadobox}[1][]{
    colback=blue!5,
    colframe=blue!60!black,
    title={\textbf{#1}},
    fonttitle=\bfseries
}

% ================== TÍTULO ==================
\title{\textbf{Lista de Exercícios}\\
\Large Autômatos Finitos Não-Determinísticos (AFN)\\
\large Resoluções com Diagramas}
\author{Linguagens Formais e Autômatos}
\date{\today}

\begin{document}

\maketitle
\tableofcontents
\newpage

% ============================================================
% EXERCÍCIO 1
% ============================================================
\section{Exercício 1: AFNs sobre $\Sigma = \{a, b, c\}$}

% ============================================================
% EXERCÍCIO 1a
% ============================================================
\subsection{Exercício 1a}

\begin{enunciadobox}[Enunciado]
$L_a = \{w \mid w \text{ contenha o sufixo } abc \text{ ou } cba\}$
\end{enunciadobox}

\begin{figure}[H]
    \centering
    \includegraphics[width=\textwidth,height=0.4\textheight,keepaspectratio]{../diagramasAFN/AFN_exe1_a.pdf}
    \caption{AFN 1a: Sufixo $abc$ ou $cba$}
\end{figure}

\newpage

% ============================================================
% EXERCÍCIO 1b
% ============================================================
\subsection{Exercício 1b}

\begin{enunciadobox}[Enunciado]
$L_b = \{w \mid w \text{ contenha pelo menos 3 ocorrências de } abc\}$
\end{enunciadobox}

\begin{figure}[H]
    \centering
    \includegraphics[width=\textwidth,height=0.4\textheight,keepaspectratio]{../diagramasAFN/AFN_exe1_b.pdf}
    \caption{AFN 1b: Pelo menos 3 ocorrências de $abc$}
\end{figure}

\newpage

% ============================================================
% EXERCÍCIO 1c
% ============================================================
\subsection{Exercício 1c}

\begin{enunciadobox}[Enunciado]
$L_c = \{w \mid \text{o último símbolo de } w \text{ seja igual ao primeiro}\}$
\end{enunciadobox}

\begin{figure}[H]
    \centering
    \includegraphics[width=\textwidth,height=0.4\textheight,keepaspectratio]{../diagramasAFN/AFN_exe1_c.pdf}
    \caption{AFN 1c: Último símbolo igual ao primeiro}
\end{figure}

\newpage

% ============================================================
% EXERCÍCIO 1d
% ============================================================
\subsection{Exercício 1d}

\begin{enunciadobox}[Enunciado]
$L_d = \{w \mid w \text{ tenha 2 } a\text{'s consecutivos ou 2 } b\text{'s consecutivos}\}$
\end{enunciadobox}

\begin{figure}[H]
    \centering
    \includegraphics[width=\textwidth,height=0.4\textheight,keepaspectratio]{../diagramasAFN/AFN_exe1_d.pdf}
    \caption{AFN 1d: Contém $aa$ ou $bb$}
\end{figure}

\newpage

% ============================================================
% EXERCÍCIO 1e
% ============================================================
\subsection{Exercício 1e}

\begin{enunciadobox}[Enunciado]
$L_e = \{w \mid aa \text{ ou } bb \text{ é subpalavra E } cccc \text{ é sufixo}\}$
\end{enunciadobox}

\begin{figure}[H]
    \centering
    \includegraphics[width=\textwidth,height=0.4\textheight,keepaspectratio]{../diagramasAFN/AFN_exe1_e.pdf}
    \caption{AFN 1e: Subpalavra $aa$ ou $bb$ E sufixo $cccc$}
\end{figure}

\newpage

% ============================================================
% EXERCÍCIO 1f
% ============================================================
\subsection{Exercício 1f}

\begin{enunciadobox}[Enunciado]
$L_f = \{w \mid w \text{ contenha as substrings } ab \text{ e } ba \text{ (em qualquer ordem)}\}$
\end{enunciadobox}

\begin{figure}[H]
    \centering
    \includegraphics[width=\textwidth,height=0.4\textheight,keepaspectratio]{../diagramasAFN/AFN_exe1_f.pdf}
    \caption{AFN 1f: Contém $ab$ E $ba$}
\end{figure}

\newpage

% ============================================================
% EXERCÍCIO 1g
% ============================================================
\subsection{Exercício 1g}

\begin{enunciadobox}[Enunciado]
$L_g = \{xyz \mid x, y, z \in \Sigma^* \text{ e } |x| = 3 \text{ e } |z| = 3\}$

(Cadeias com comprimento $\geq 6$)
\end{enunciadobox}

\begin{figure}[H]
    \centering
    \includegraphics[width=\textwidth,height=0.4\textheight,keepaspectratio]{../diagramasAFN/AFN_exe1_g.pdf}
    \caption{AFN 1g: Comprimento $\geq 6$}
\end{figure}

\newpage

% ============================================================
% EXERCÍCIO 1h
% ============================================================
\subsection{Exercício 1h}

\begin{enunciadobox}[Enunciado]
$L_h = \{w \mid w \text{ contenha exatamente um } a\}$
\end{enunciadobox}

\begin{figure}[H]
    \centering
    \includegraphics[width=\textwidth,height=0.4\textheight,keepaspectratio]{../diagramasAFN/AFN_exe1_h.pdf}
    \caption{AFN 1h: Exatamente um $a$}
\end{figure}

\newpage

% ============================================================
% EXERCÍCIO 1i
% ============================================================
\subsection{Exercício 1i}

\begin{enunciadobox}[Enunciado]
$L_i = \{w \mid \#a + \#b + 2 \cdot \#c \equiv 0 \pmod{6}\}$

(Quantidade de $a$'s + quantidade de $b$'s + dobro de $c$'s divisível por 6)
\end{enunciadobox}

\begin{figure}[H]
    \centering
    \includegraphics[width=\textwidth,height=0.4\textheight,keepaspectratio]{../diagramasAFN/AFN_exe1_i.pdf}
    \caption{AFN 1i: Soma ponderada divisível por 6}
\end{figure}

\newpage

% ============================================================
% EXERCÍCIO 1j
% ============================================================
\subsection{Exercício 1j}

\begin{enunciadobox}[Enunciado]
$L_j = \{w \mid w \text{ contenha um número par de substrings } ba\}$
\end{enunciadobox}

\begin{figure}[H]
    \centering
    \includegraphics[width=\textwidth,height=0.4\textheight,keepaspectratio]{../diagramasAFN/AFN_exe1_j.pdf}
    \caption{AFN 1j: Número par de substrings $ba$}
\end{figure}

\newpage

% ============================================================
% EXERCÍCIO 2
% ============================================================
\section{Exercício 2: Último símbolo repetido}

\begin{enunciadobox}[Enunciado]
Dado $\Sigma = \{1, 2, 3\}$, construa um AFN para:

$L = \{w \mid \text{o último símbolo de } w \text{ aparece pelo menos duas vezes, porém nenhum símbolo maior aparece entre as duas últimas ocorrências de tal símbolo}\}$
\end{enunciadobox}

\begin{figure}[H]
    \centering
    \includegraphics[width=\textwidth,height=0.45\textheight,keepaspectratio]{../diagramasAFN/AFN_exe2.pdf}
    \caption{AFN 2: Último símbolo repetido sem maior entre ocorrências}
\end{figure}

\newpage

% ============================================================
% EXERCÍCIO 3
% ============================================================
\section{Exercício 3: Condição complexa com sufixos e subpalavras}

\begin{enunciadobox}[Enunciado]
Dado $\Sigma = \{0, 1\}$, construa um AFN para:

$L = \{w \mid (w \text{ termina em } 01 \land w \text{ contém } 011) \lor (w \text{ termina em } 10 \land w \text{ contém } 100)\}$
\end{enunciadobox}

\begin{figure}[H]
    \centering
    \includegraphics[width=\textwidth,height=0.45\textheight,keepaspectratio]{../diagramasAFN/AFN_exe3.pdf}
    \caption{AFN 3: Condição complexa com sufixos e subpalavras}
\end{figure}

\newpage

% ============================================================
% EXERCÍCIO 4: Expressões Regulares
% ============================================================
\section{Exercício 4: AFNs a partir de Expressões Regulares}

% ============================================================
% EXERCÍCIO 4a
% ============================================================
\subsection{Exercício 4a}

\begin{enunciadobox}[Enunciado]
Expressão Regular: $(a \cup b \cup c)^* c$

Linguagem: Cadeias sobre $\{a,b,c\}$ que terminam com $c$
\end{enunciadobox}

\begin{figure}[H]
    \centering
    \includegraphics[width=\textwidth,height=0.4\textheight,keepaspectratio]{../diagramasAFN/AFN_exe4_a.pdf}
    \caption{AFN 4a: $(ab)^*$}
\end{figure}

\newpage

% ============================================================
% EXERCÍCIO 4b
% ============================================================
\subsection{Exercício 4b}

\begin{enunciadobox}[Enunciado]
Expressão Regular: $(ab)^+$

Linguagem: Uma ou mais repetições de $ab$
\end{enunciadobox}

\begin{figure}[H]
    \centering
    \includegraphics[width=\textwidth,height=0.4\textheight,keepaspectratio]{../diagramasAFN/AFN_exe4_b.pdf}
    \caption{AFN 4b: $a^*b^*c^*$}
\end{figure}

\newpage

% ============================================================
% EXERCÍCIO 4c
% ============================================================
\subsection{Exercício 4c}

\begin{enunciadobox}[Enunciado]
Expressão Regular: $(aa^* \cup bb^*)$

Linguagem: Sequência de um ou mais $a$'s OU um ou mais $b$'s
\end{enunciadobox}

\begin{figure}[H]
    \centering
    \includegraphics[width=\textwidth,height=0.4\textheight,keepaspectratio]{../diagramasAFN/AFN_exe4_c.pdf}
    \caption{AFN 4c: $a(a \cup b)^*b$}
\end{figure}

\newpage

% ============================================================
% EXERCÍCIO 4d
% ============================================================
\subsection{Exercício 4d}

\begin{enunciadobox}[Enunciado]
Expressão Regular: $a(b \cup a)^* b$

Linguagem: Começa com $a$, termina com $b$, qualquer coisa no meio
\end{enunciadobox}

\begin{figure}[H]
    \centering
    \includegraphics[width=\textwidth,height=0.4\textheight,keepaspectratio]{../diagramasAFN/AFN_exe4_d.pdf}
    \caption{AFN 4d: $a^+b^+c^+$}
\end{figure}

\newpage

% ============================================================
% EXERCÍCIO 4e
% ============================================================
\subsection{Exercício 4e}

\begin{enunciadobox}[Enunciado]
Expressão Regular: $(aa \cup bb)^* c (a \cup b)^*$

Linguagem: Pares de $aa$ ou $bb$, seguidos de $c$, seguidos de qualquer coisa
\end{enunciadobox}

\begin{figure}[H]
    \centering
    \includegraphics[width=\textwidth,height=0.4\textheight,keepaspectratio]{../diagramasAFN/AFN_exe4_e.pdf}
    \caption{AFN 4e: $((a \cup b)c)^*$}
\end{figure}

\newpage

% ============================================================
% EXERCÍCIO 4f
% ============================================================
\subsection{Exercício 4f}

\begin{enunciadobox}[Enunciado]
Expressão Regular: $(a \cup b)^* (bc \cup cb) (a \cup b)^+$

Linguagem: Contém $bc$ ou $cb$ com pelo menos um símbolo depois
\end{enunciadobox}

\begin{figure}[H]
    \centering
    \includegraphics[width=\textwidth,height=0.4\textheight,keepaspectratio]{../diagramasAFN/AFN_exe4_f.pdf}
    \caption{AFN 4f: $(a \cup b)^*c(a \cup b)^*$}
\end{figure}

\newpage

% ============================================================
% EXERCÍCIO 4g
% ============================================================
\subsection{Exercício 4g}

\begin{enunciadobox}[Enunciado]
Expressão Regular: $a^* ((b^+c^*) \cup (b^+a^*))$

Linguagem: Zero ou mais $a$'s, seguido de $b$'s com $c$'s ou $a$'s opcionais
\end{enunciadobox}

\begin{figure}[H]
    \centering
    \includegraphics[width=\textwidth,height=0.4\textheight,keepaspectratio]{../diagramasAFN/AFN_exe4_g.pdf}
    \caption{AFN 4g: $((ab \cup ba)c)^*$}
\end{figure}

\newpage

% ============================================================
% EXERCÍCIO 4h
% ============================================================
\subsection{Exercício 4h}

\begin{enunciadobox}[Enunciado]
Expressão Regular: $(a \cup b)^* \cup (b \cup c)^* \cup (a \cup c)^* \cup (b \cup c)^*$

Linguagem: União de linguagens com pares de símbolos
\end{enunciadobox}

\begin{figure}[H]
    \centering
    \includegraphics[width=\textwidth,height=0.4\textheight,keepaspectratio]{../diagramasAFN/AFN_exe4_h.pdf}
    \caption{AFN 4h: $(a \cup b \cup c)^*(abc \cup cba)(a \cup b \cup c)^*$}
\end{figure}

\newpage

% ============================================================
% EXERCÍCIO 4i
% ============================================================
\subsection{Exercício 4i}

\begin{enunciadobox}[Enunciado]
Expressão Regular: $((a \cup b)^* \cup (b \cup c)^* \cup (a \cup c)^* \cup (b \cup c)^*)^+$

Linguagem: Uma ou mais repetições da união anterior
\end{enunciadobox}

\begin{figure}[H]
    \centering
    \includegraphics[width=\textwidth,height=0.4\textheight,keepaspectratio]{../diagramasAFN/AFN_exe4_i.pdf}
    \caption{AFN 4i: $(abc)^* \cup (cba)^*$}
\end{figure}

\newpage

% ============================================================
% EXERCÍCIO 4j
% ============================================================
\subsection{Exercício 4j}

\begin{enunciadobox}[Enunciado]
Expressão Regular: $(a \cup b)^* aa (a \cup b)^*$

Linguagem: Contém $aa$ como substring
\end{enunciadobox}

\begin{figure}[H]
    \centering
    \includegraphics[width=\textwidth,height=0.4\textheight,keepaspectratio]{../diagramasAFN/AFN_exe4_j.pdf}
    \caption{AFN 4j: $a(a \cup b)^*a \cup b(a \cup b)^*b \cup a \cup b$}
\end{figure}

\newpage

% ============================================================
% RESUMO
% ============================================================
\section{Resumo dos Exercícios}

\begin{center}
\begin{tabular}{|c|l|c|}
\hline
\textbf{Ex.} & \textbf{Linguagem} & \textbf{Alfabeto} \\
\hline
1a & Sufixo $abc$ ou $cba$ & $\{a,b,c\}$ \\
1b & Pelo menos 3 ocorrências de $abc$ & $\{a,b,c\}$ \\
1c & Primeiro símbolo = último símbolo & $\{a,b,c\}$ \\
1d & Contém $aa$ ou $bb$ & $\{a,b,c\}$ \\
1e & ($aa$ ou $bb$) E sufixo $cccc$ & $\{a,b,c\}$ \\
1f & Contém $ab$ E $ba$ & $\{a,b,c\}$ \\
1g & Comprimento $\geq 6$ & $\{a,b,c\}$ \\
1h & Exatamente um $a$ & $\{a,b,c\}$ \\
1i & $\#a + \#b + 2\#c \equiv 0 \pmod{6}$ & $\{a,b,c\}$ \\
1j & Número par de substrings $ba$ & $\{a,b,c\}$ \\
\hline
2 & Último símbolo repetido (sem maior entre) & $\{1,2,3\}$ \\
\hline
3 & (Termina 01 $\land$ contém 011) $\lor$ (Termina 10 $\land$ contém 100) & $\{0,1\}$ \\
\hline
4a & $(ab)^*$ & $\{a,b\}$ \\
4b & $a^*b^*c^*$ & $\{a,b,c\}$ \\
4c & $a(a \cup b)^*b$ & $\{a,b\}$ \\
4d & $a^+b^+c^+$ & $\{a,b,c\}$ \\
4e & $((a \cup b)c)^*$ & $\{a,b,c\}$ \\
4f & $(a \cup b)^*c(a \cup b)^*$ & $\{a,b,c\}$ \\
4g & $((ab \cup ba)c)^*$ & $\{a,b,c\}$ \\
4h & $(a \cup b \cup c)^*(abc \cup cba)(a \cup b \cup c)^*$ & $\{a,b,c\}$ \\
4i & $(abc)^* \cup (cba)^*$ & $\{a,b,c\}$ \\
4j & $a(a \cup b)^*a \cup b(a \cup b)^*b \cup a \cup b$ & $\{a,b\}$ \\
\hline
\end{tabular}
\end{center}

\section*{Observações}

\begin{itemize}
    \item Todos os AFNs foram validados com 500 testes aleatórios cada
    \item Os diagramas foram gerados automaticamente usando Mermaid
    \item Os arquivos JSON de definição estão em \texttt{inputAFN/}
    \item Os diagramas PDF estão em \texttt{diagramasAFN/}
    \item AFNs podem usar transições-$\varepsilon$ para simplificar construção
\end{itemize}

\end{document}
